\documentclass[leqno]{article}

\usepackage[brazil]{babel} %\usepackage[latin1]{inputenc}
\usepackage{a4wide}
\setlength{\oddsidemargin}{-0.2in}
% % \setlength{\oddsidemargin}{0.2in}
\setlength{\evensidemargin}{-0.2in}
% % \setlength{\evensidemargin}{0.5in}
% % \setlength{\textwidth}{5.5in}
\setlength{\textwidth}{6.5in}
\setlength{\topmargin}{-1.2in}
\setlength{\textheight}{10in}
\usepackage[]{amsfonts} \usepackage[]{amsmath}
\usepackage[]{amssymb} \usepackage[]{latexsym}
\usepackage{graphicx,color} \usepackage{amsthm}
\usepackage{mathrsfs} \usepackage{url}
\usepackage{cancel} \usepackage{enumerate}
\usepackage{xifthen} \usepackage{tikz}
\usetikzlibrary{automata,arrows,positioning,calc}

\numberwithin{equation}{section}

\setlength{\parindent}{12 pt}

\begin{document}
	
	\newtheorem{teo}{Teorema}[section] \newtheorem*{teo*}{Teorema}
	\newtheorem{prop}[teo]{Proposição} \newtheorem*{prop*}{Proposição}
	\newtheorem{lema}[teo]{Lemma} \newtheorem*{lema*}{Lema}
	\newtheorem{cor}[teo]{Corolário} \newtheorem*{cor*}{Corolário}
	
	\theoremstyle{definition}
	\newtheorem{defi}[teo]{Definição} \newtheorem*{defi*}{Definição}
	\newtheorem{exem}[teo]{Exemplo} \newtheorem*{exem*}{Exemplo}
	\newtheorem{obs}[teo]{Observação} \newtheorem*{obs*}{Observação}
	\newtheorem*{hipo}{Hipóteses}
	\newtheorem*{nota}{Notação}
	
	\newcommand{\ds}{\displaystyle} \newcommand{\nl}{\newline}
	\newcommand{\eps}{\varepsilon} \newcommand{\ssty}{\scriptstyle}
	\newcommand{\bE}{\mathbb{E}}
	\newcommand{\cB}{\mathcal{B}}
	\newcommand{\cF}{\mathcal{F}}
	\newcommand{\cA}{\mathcal{A}}
	\newcommand{\cM}{\mathcal{M}}
	\newcommand{\cD}{\mathcal{D}}
	\newcommand{\cN}{\mathcal{N}}
	\newcommand{\cL}{\mathcal{L}}
	\newcommand{\cLN}{\mathcal{LN}}
	\newcommand{\bP}{\mathbb{P}}
	\newcommand{\bQ}{\mathbb{Q}}
	\newcommand{\bN}{\mathbb{N}}
	\newcommand{\bR}{\mathbb{R}}
	\newcommand{\bZ}{\mathbb{Z}}
	
	\newcommand{\bfw}{\mathbf{w}}
	\newcommand{\bfv}{\mathbf{v}}
	\newcommand{\bfu}{\mathbf{u}}
	\newcommand{\bfx}{\mathbf{x}}
	\newcommand{\bfb}{\mathbf{b}}
	
	\newcommand{\bvecc}[2]{%
		\begin{bmatrix} #1 \\ #2  \end{bmatrix}
	}
	\newcommand{\bveccc}[3]{%
		\begin{bmatrix} #1 \\ #2 \\ #3  \end{bmatrix}
	}
	
	\newenvironment{sol} 
	{
		\vspace{4mm}
		\noindent\textbf{Resolução:}
		\strut\newline
		\smallskip
		\hspace{-3.5mm} 
	} 
	% Objetos que aparecem *após* o ambiente. 
	% (você pode, por exemplo, modificar, 
	% ou remover, a barra horizontal} 
	%{\noindent\rule{4cm}{.1mm}}
	
	
	\title{Álgebra Linear - Lista de Exercícios 8}
	
	\author{Iara Cristina Mescua Castro}
	
	\date{\today}
	
	\maketitle
	
	\begin{enumerate}
		
		%%%%%%%%%%%%%%%%%%%%%%%%%%%%%%%%%%%%%%%%%%%%%%%%%%%%%%%%%
		%%%%%%%%%%%%%%%%%%%%%% Exercício 1 %%%%%%%%%%%%%%%%%%%%%%
		%%%%%%%%%%%%%%%%%%%%%%%%%%%%%%%%%%%%%%%%%%%%%%%%%%%%%%%%%
		
		\item Escreva as 3 equações para a reta $b = C + Dt$ passar pelos pontos $(-1,7)$, $(1,7)$, $(2,21)$. Ache a solução de mínimos quadrados $\hat{x}$ e a projeção $p = A\hat{x}$.
		
		\begin{sol}
			
			$$\begin{bmatrix}
				1 & -1 \\
				1 & 1 \\
				1 & 2
			\end{bmatrix}\begin{bmatrix}
			C \\
			D
			\end{bmatrix} = \begin{bmatrix}
			7 \\
			7 \\
			21
			\end{bmatrix}$$
			
		$
		\left\{
		\begin {array}{cl}
		C - D = 7\\
		C + D = 7\\
		C + 2D = 21
		\end{array}
		\right.
		$\\
		
		Para a solução dos mínimos quadrados $\hat{x}$:
		
		$$\begin{bmatrix}
			1 & 1 & 1\\
			-1 & 1 & 2
		\end{bmatrix}\begin{bmatrix}
			1 & -1 \\
			1 & 1 \\
			1 & 2
		\end{bmatrix}\begin{bmatrix}
		C \\
		D
		\end{bmatrix} = \begin{bmatrix}
		1 & 1 & 1\\
		-1 & 1 & 2
		\end{bmatrix} \begin{bmatrix}
		7 \\
		7 \\
		21
		\end{bmatrix}$$
	
		$$\begin{bmatrix}
			3 & 2\\
			2 & 6
		\end{bmatrix}\begin{bmatrix}
		C \\
		D
		\end{bmatrix} = \begin{bmatrix}
		35\\
		42
		\end{bmatrix}$$
	
		$
		\left\{
		\begin {array}{cl}
		3C + 2D = 35\\
		2C + 6D = 42
		\end{array}
		\right.
		$\\
		
		$C = 9$ e $D = 4$, então: $\hat{x} = \begin{bmatrix}
			9\\
			4
		\end{bmatrix}$
	
		$p = A\hat{x} = \begin{bmatrix}
			1 & -1 \\
			1 & 1 \\
			1 & 2
		\end{bmatrix}\begin{bmatrix}
		9\\
		4
		\end{bmatrix} = \begin{bmatrix}
		5\\
		13\\
		17
		\end{bmatrix}$
		\end{sol}
		
		%%%%%%%%%%%%%%%%%%%%%%%%%%%%%%%%%%%%%%%%%%%%%%%%%%%%%%%%%
		%%%%%%%%%%%%%%%%%%%%%% Exercício 2 %%%%%%%%%%%%%%%%%%%%%%
		%%%%%%%%%%%%%%%%%%%%%%%%%%%%%%%%%%%%%%%%%%%%%%%%%%%%%%%%%
		
		\item Dado o problema acima, quais dos quatro subespaços fundamentais contêm o vetor erro $e = b - p$? E o vetor $p$? E o vetor $\hat{x}$? Qual é o núcleo de $A$?
		
		\begin{sol}
			$e = b - p =\begin{bmatrix}
				2 \\
				-6 \\
				4
			\end{bmatrix}$
			, uma vez que o vetor de erro $e$ é ortogonal ao espaço da coluna C(A), ele está no
			o espaço nulo esquerdo $N(A^T)$.\\
			A projeção $p$ está em $C(A)$, porque é a projeção no espaço coluna.\\
			Uma vez que o espaço de linha $C(A^T)$ e o espaço nulo $N(A)$ abrangem todo o espaço, e
			sempre podemos modificar o vetor $\hat{x}$ por um vetor em $N(A)$ (o que não afeta a
			projeção $A\hat{x}$). Portanto, podemos escolher $\hat{x}$ para estar no espaço de linha $C(A^T)$.\\
			$N(A) = \{0\}$, contém apenas o vetor nulo.
			
		\end{sol}
		
		%%%%%%%%%%%%%%%%%%%%%%%%%%%%%%%%%%%%%%%%%%%%%%%%%%%%%%%%%
		%%%%%%%%%%%%%%%%%%%%%% Exercício 3 %%%%%%%%%%%%%%%%%%%%%%
		%%%%%%%%%%%%%%%%%%%%%%%%%%%%%%%%%%%%%%%%%%%%%%%%%%%%%%%%%
		
		\item Ache a melhor reta que se ajusta aos pontos $t = -2, -1, 0, 1, 2$ e $b = 4, 2, -1, 0, 0$.
		
		\begin{sol}
			
			$$C + Dt = \begin{bmatrix}
				1 & -2 \\
				1 & -1 \\
				1 & 0 \\
				1 & 1 \\ 
				1 & 2
			\end{bmatrix}\begin{bmatrix}
				C \\
				D
			\end{bmatrix} = \begin{bmatrix}
				4 \\
				2 \\
				-1 \\ 
				0 \\
				0
			\end{bmatrix}$$
		
		$$\begin{bmatrix}
			-2 & -1 & 0 & 1 & 2\\
			1 & 1 & 1 & 1 & 1
		\end{bmatrix}\begin{bmatrix}
			1 & -2 \\
			1 & -1 \\
			1 & 0 \\
			1 & 1 \\ 
			1 & 2
		\end{bmatrix}\begin{bmatrix}
			C \\
			D
		\end{bmatrix} = \begin{bmatrix}
		-2 & -1 & 0 & 1 & 2\\
		1 & 1 & 1 & 1 & 1
		\end{bmatrix}\begin{bmatrix}
			4 \\
			2 \\
			-1 \\ 
			0 \\
			0
		\end{bmatrix}$$
	
		$$\begin{bmatrix}
			10 & 0 \\
			0 & 5
		\end{bmatrix}\begin{bmatrix}
		C \\
		D
		\end{bmatrix} = \begin{bmatrix}
		-10 \\
		5
		\end{bmatrix}$$
		
		
		$
		\left\{
		\begin {array}{cl}
		10C = -10 \rightarrow C = -1\\
		5D = 5 \rightarrow D = 1
		\end{array}
		\right.
		$\\
		
		A melhor reta que se ajusta a esses pontos é: $b = 1 - t$
		\end{sol}
		
		%%%%%%%%%%%%%%%%%%%%%%%%%%%%%%%%%%%%%%%%%%%%%%%%%%%%%%%%%
		%%%%%%%%%%%%%%%%%%%%%% Exercício 4 %%%%%%%%%%%%%%%%%%%%%%
		%%%%%%%%%%%%%%%%%%%%%%%%%%%%%%%%%%%%%%%%%%%%%%%%%%%%%%%%%
		
		\item Dados os vetores
		$$v_1 = [1 \ -1 \ 0 \ 0], \ v_2 = [0 \ 1 \ -1 \ 0] \mbox{ e } v_3 = [0 \ 0 \ 1 \ -1],$$
		use o método de Gram-Schmidt para achar uma base ortonormal que gera o mesmo espaço de $v_1, v_2, v_3$.
		
		\begin{sol}
			
			$u_1 = v_1 = (1 \ -1 \ 0 \ 0)$\\
			$u_2 = v_2 - \frac{(u_1^Tv_2)}{u_1^Tu_1}u_1 = (0 \ 1 \ -1 \ 0) + \frac{1}{2}(1 \ -1 \ 0 \ 0) = (\frac{1}{2} \ \frac{1}{2} \ -1 \ 0)$\\
			$u_3 = v_3 - \frac{(u_1^Tv_3)}{u_1^Tu_1}u_1 - \frac{(u_2^Tv_2)}{u_2^Tu_2}u_2 = (0 \ 0 \ 1 \ -1) + \frac{2}{3}(\frac{1}{2} \ \frac{1}{2} \ -1 \ 0) = (\frac{1}{3} \ \frac{1}{3} \ \frac{1}{3} \ 0)$
			
		\end{sol}
		
		%%%%%%%%%%%%%%%%%%%%%%%%%%%%%%%%%%%%%%%%%%%%%%%%%%%%%%%%%
		%%%%%%%%%%%%%%%%%%%%%% Exercício 5 %%%%%%%%%%%%%%%%%%%%%%
		%%%%%%%%%%%%%%%%%%%%%%%%%%%%%%%%%%%%%%%%%%%%%%%%%%%%%%%%%
		
		\item Se os elementos de cada linha de uma matriz $A$ somam zero, ache uma solução para $Ax = 0$ e conclua que $\det A = 0$. Se esses elementos somam 1, conclua que $\det(A - I) = 0$.
		
		\begin{sol}
			Supondo que x seja uma matriz:
			$\begin{bmatrix}
				1 \\
				1 \\
				1 \\
				\vdots
			\end{bmatrix}$, ao fazer $Ax$, seus componentes serão a soma das linhas de A. Visto que os elementos de cada linha de uma matriz $A$ somam zero, então Ax = 0. Como A tem espaço nulo diferente de zero, ele não é invertível e $det A = 0$.
		
			Se os elementos de cada linha de A somam 1, então os elementos de cada
			linha de (A - I) soma zero. Portanto, (A - I) tem um espaço nulo diferente de zero e $det (A - I) = 0$.
			
		\end{sol}
		
		%%%%%%%%%%%%%%%%%%%%%%%%%%%%%%%%%%%%%%%%%%%%%%%%%%%%%%%%%
		%%%%%%%%%%%%%%%%%%%%%% Exercício 6 %%%%%%%%%%%%%%%%%%%%%%
		%%%%%%%%%%%%%%%%%%%%%%%%%%%%%%%%%%%%%%%%%%%%%%%%%%%%%%%%%
		
		\item Use as propriedades do determinante (e não suas fórmulas) para mostrar que
		$$\det \begin{bmatrix} 1 & a & a^2 \\ 1 & b & b^2 \\ 1 & c & c^2 \end{bmatrix} = (b-a)(c-a)(c-b).$$
		
		\begin{sol}
			Pela propriedade: $\begin{bmatrix}
				a & b \\
				c & d \\
			\end{bmatrix} =\begin{bmatrix}
			a - \alpha c & b - \alpha d\\
			c & d \\
			\end{bmatrix} $
			
			$(L_2 - L_1)$
			$(L_3 - L_2)$
			
		$$\det \begin{bmatrix}
			 1 & a & a^2 \\
			  1 & b & b^2 \\
			   1 & c & c^2 \end{bmatrix} = 
		   \det \begin{bmatrix}
		   	1 & a & a^2 \\
		   	1 & b-a & b^2 - a^2 \\
		   	1 & c-b & c^2 - b^2
	   		\end{bmatrix} = 
	   	1 \begin{bmatrix}
	   		b-a & b^2 - a^2 \\
	   		c-b & c^2 - b^2 \\
		  \end{bmatrix}$$
	  $$= (b - a)(c^2 - b^2) - (b^2 - a^2)(c-b)$$
	  $$= (b - a)(c - b)(c + b) - (b - a)(b + a)(c - b)$$
	  $$= (b - a)((c - b)(c + b)-(b + a)(c - b))$$
	  $$= (b - a)(c - b)( c + b - b - a)$$
	  $$= (b - a)(c - b)(c - a)$$
	  $$= (b-a)(c-a)(c-b)$$
		\end{sol}
		
		%%%%%%%%%%%%%%%%%%%%%%%%%%%%%%%%%%%%%%%%%%%%%%%%%%%%%%%%%
		%%%%%%%%%%%%%%%%%%%%%% Exercício 7 %%%%%%%%%%%%%%%%%%%%%%
		%%%%%%%%%%%%%%%%%%%%%%%%%%%%%%%%%%%%%%%%%%%%%%%%%%%%%%%%%
		
		\item Calcule
		$$\det \begin{bmatrix} 0 & 0 & 0 & 1 \\ 1 & 0 & 0 & 0 \\ 0 & 1 & 0 & 0 \\ 0 & 0 & 1 & 0  \end{bmatrix}.$$
		
		\begin{sol}
			$DetA = -1$
			$$DetA = Det\begin{bmatrix}
			0 & 0 & 0 & 1 \\
			1 & 0 & 0 & 0 \\
			0 & 1 & 0 & 0 \\
			0 & 0 & 1 & 0  
		    \end{bmatrix}$$ $$= -Det\begin{bmatrix}
		    1 & 0 & 0 & 0 \\
		    0 & 0 & 0 & 1 \\
		    0 & 1 & 0 & 0 \\
		    0 & 0 & 1 & 0  
	   		\end{bmatrix} (L_1 \leftrightarrow L_2) = Det\begin{bmatrix}
	   		1 & 0 & 0 & 0 \\
	   		0 & 1 & 0 & 0 \\
	   		0 & 0 & 0 & 1 \\
	   		0 & 0 & 1 & 0  
	   		\end{bmatrix} (L_2 \leftrightarrow L_3) = -Det\begin{bmatrix}
	   		1 & 0 & 0 & 0 \\
	   		0 & 1 & 0 & 0 \\
	   		0 & 0 & 1 & 0 \\
	   		0 & 0 & 0 & 1  
	   		\end{bmatrix} (L_3 \leftrightarrow L_4) = -1$$
			
		\end{sol}
		
		%%%%%%%%%%%%%%%%%%%%%%%%%%%%%%%%%%%%%%%%%%%%%%%%%%%%%%%%%
		%%%%%%%%%%%%%%%%%%%%%% Exercício 8 %%%%%%%%%%%%%%%%%%%%%%
		%%%%%%%%%%%%%%%%%%%%%%%%%%%%%%%%%%%%%%%%%%%%%%%%%%%%%%%%%
		
		\item Use o fato de que
		$$\det \begin{bmatrix} 1 & 1 & 1 & 1 \\ 1 & 2 & 3 & 4 \\ 1 & 3 & 6 & 10 \\ 1 & 4 & 10 & 20  \end{bmatrix} = 1$$
		para mostrar que
		
		$$\det \begin{bmatrix} 1 & 1 & 1 & 1 \\ 1 & 2 & 3 & 4 \\ 1 & 3 & 6 & 10 \\ 1 & 4 & 10 & \mathbf{19}  \end{bmatrix} = 0.$$
		
		\begin{sol}
			$$\det \begin{bmatrix}
			1 & 1 & 1 & 1 \\
			1 & 2 & 3 & 4 \\
			1 & 3 & 6 & 10 \\
			1 & 4 & 10 & 19
		  \end{bmatrix} = 1 cof(1) + 4 cof(4) - 10 cof(10) + 19 cof(19)$$
	   Visto que $20 cof(20) = 20 cof(19)$:
		$$= 1 cof(1) + 4 cof(4) - 10 cof(10) + 20 cof(20) - 1 cof(20)$$
		$$\det \begin{bmatrix}
			1 & 1 & 1 & 1 \\
			1 & 2 & 3 & 4 \\
			1 & 3 & 6 & 10 \\
			1 & 4 & 10 & 20 
		\end{bmatrix} = 1 cof(1) + 4 cof(4) - 10 cof(10) + 20 cof(20) = 1$$
		Substituindo:
		$$\det \begin{bmatrix}
			1 & 1 & 1 & 1 \\
			1 & 2 & 3 & 4 \\
			1 & 3 & 6 & 10 \\
			1 & 4 & 10 & 19
		\end{bmatrix} = 1 - 1 cof(20)$$
		
		$$1 cof(20) = \det \begin{bmatrix}
			1 & 1 & 1  \\
			1 & 2 & 3  \\
			1 & 3 & 6 
		\end{bmatrix}$$
	
		Sabendo que essa matriz é uma matriz de Pascal simétrica, ela possui determinante igual a 1, então:
		$$\det \begin{bmatrix}
			1 & 1 & 1 & 1 \\
			1 & 2 & 3 & 4 \\
			1 & 3 & 6 & 10 \\
			1 & 4 & 10 & 19
		\end{bmatrix} = 1 - 1 = 0$$
		\end{sol}
		
		%%%%%%%%%%%%%%%%%%%%%%%%%%%%%%%%%%%%%%%%%%%%%%%%%%%%%%%%%
		%%%%%%%%%%%%%%%%%%%%%% Exercício 9 %%%%%%%%%%%%%%%%%%%%%%
		%%%%%%%%%%%%%%%%%%%%%%%%%%%%%%%%%%%%%%%%%%%%%%%%%%%%%%%%%
		
		\item Ache o determinante da seguinte matriz:
		$$A = \begin{bmatrix} 1 & 1 & 4 \\ 1 & 2 & 2 \\ 1 & 2 & 5   \end{bmatrix}$$
		usando cofatores. O que acontece quando mudamos o valor 4 para 100?
		
		\begin{sol}
			Pela regra dos cofatores:\\
			$$
			detA = det\begin{bmatrix}
			 1 & 1 & 4 \\
			 1 & 2 & 2 \\ 
			 1 & 2 & 5
		    \end{bmatrix} = 
	    	1 \cdot det\begin{bmatrix}
	    		2 & 2 \\
	    		2 & 5
	    	\end{bmatrix}
			- 1 \cdot det\begin{bmatrix}
				1 & 2 \\
				1 & 5
			\end{bmatrix}
			 + 4 \cdot det\begin{bmatrix}
			 	1 & 2 \\
			 	1 & 2
			 \end{bmatrix}$$
		 
		 $$detA = 6 - 3 + 0$$
		 $$detA = 3$$
		 
		 Caso mudemos o valor de 4 para 100, a determinante continua a mesma, pois $det\begin{bmatrix}
		 	1 & 2 \\
		 	1 & 2
		 \end{bmatrix} = 0$, então poderia ser qualquer valor que a determinante de A continuaria 3.  
		\end{sol}
		
	\end{enumerate}
\end{document}