\documentclass[leqno]{article}

\usepackage[brazil]{babel} \usepackage[utf8]{inputenc}
\usepackage{a4wide}
\setlength{\oddsidemargin}{-0.2in}
% % \setlength{\oddsidemargin}{0.2in}
\setlength{\evensidemargin}{-0.2in}
% % \setlength{\evensidemargin}{0.5in}
% % \setlength{\textwidth}{5.5in}
\setlength{\textwidth}{6.5in}
\setlength{\topmargin}{-1.2in}
\setlength{\textheight}{10in}
\usepackage[]{amsfonts} \usepackage[]{amsmath}
\usepackage[]{amssymb} \usepackage[]{latexsym}
\usepackage{graphicx,color} \usepackage{amsthm}
\usepackage{mathrsfs} \usepackage{url}
\usepackage{cancel} \usepackage[inline]{enumitem}
\usepackage{xifthen} \usepackage{tikz}
\usetikzlibrary{automata,arrows,positioning,calc}

\numberwithin{equation}{section}

\setlength{\parindent}{12 pt}

\newenvironment{sol}
{
	\vspace{4mm}
	\noindent\textbf{Resolução:}
	\strut\newline
	\smallskip
	\hspace{-3.5mm}
}
% Objetos que aparecem *após* o ambiente; 
% nestas configurações, estamos desenhando uma 
% linha horizontal. 
% (você pode, por exemplo, modificar 
% ou remover este elemento gráfico) 
%{\noindent\rule{4cm}{.1mm}}

\begin{document}
	
	\newtheorem{teo}{Teorema}[section] \newtheorem*{teo*}{Teorema}
	\newtheorem{prop}[teo]{Proposição} \newtheorem*{prop*}{Proposição}
	\newtheorem{lema}[teo]{Lemma} \newtheorem*{lema*}{Lema}
	\newtheorem{cor}[teo]{Corolário} \newtheorem*{cor*}{Corolário}
	
	\theoremstyle{definition}
	\newtheorem{defi}[teo]{Definição} \newtheorem*{defi*}{Definição}
	\newtheorem{exem}[teo]{Exemplo} \newtheorem*{exem*}{Exemplo}
	\newtheorem{obs}[teo]{Observação} \newtheorem*{obs*}{Observação}
	\newtheorem*{hipo}{Hipóteses}
	\newtheorem*{nota}{Notação}
	
	\newcommand{\ds}{\displaystyle} \newcommand{\nl}{\newline}
	\newcommand{\eps}{\varepsilon} \newcommand{\ssty}{\scriptstyle}
	\newcommand{\bE}{\mathbb{E}}
	\newcommand{\cB}{\mathcal{B}}
	\newcommand{\cF}{\mathcal{F}}
	\newcommand{\cA}{\mathcal{A}}
	\newcommand{\cM}{\mathcal{M}}
	\newcommand{\cD}{\mathcal{D}}
	\newcommand{\cN}{\mathcal{N}}
	\newcommand{\cL}{\mathcal{L}}
	\newcommand{\cLN}{\mathcal{LN}}
	\newcommand{\bP}{\mathbb{P}}
	\newcommand{\bQ}{\mathbb{Q}}
	\newcommand{\bN}{\mathbb{N}}
	\newcommand{\bR}{\mathbb{R}}
	\newcommand{\bZ}{\mathbb{Z}}
	
	\newcommand{\bfw}{\mathbf{w}}
	\newcommand{\bfv}{\mathbf{v}}
	\newcommand{\bfu}{\mathbf{u}}
	\newcommand{\bfx}{\mathbf{x}}
	\newcommand{\bfb}{\mathbf{b}}
	
	\newcommand{\bvecc}[2]{%
		\begin{bmatrix} #1 \\ #2  \end{bmatrix}
	}
	\newcommand{\bveccc}[3]{%
		\begin{bmatrix} #1 \\ #2 \\ #3  \end{bmatrix}
	}
	
	
	\title{Álgebra Linear - Lista de Exercícios 11 - Simulado}
	
	\author{Iara Cristina Mescua Castro} 
	
	\date{}
	
	\maketitle
	
	\begin{enumerate}
		
		%%%%%%%%%%%%%%%%%%%%%%%%%%%%%%%%%%%%%%%%%%%%%%%%%%%%%%%%%
		%%%%%%%%%%%%%%%%%%%%%% Exercício 1 %%%%%%%%%%%%%%%%%%%%%%
		%%%%%%%%%%%%%%%%%%%%%%%%%%%%%%%%%%%%%%%%%%%%%%%%%%%%%%%%%
		
		\item Verdadeiro ou falso (prove ou dê um contra-exemplo):
		
		\begin{enumerate}
			
			\item Se $A$ é singular, então $AB$ também é singular. 
			
			\item O determinante de $A$ é sempre o produto de seus pivôs.
			
			\item O determinante de $A - B$ é $\det A - \det B$.
			
			\item $AB$ e $BA$ tem o mesmo determinante. 	
			 
		\end{enumerate}
		
		\begin{sol} 
			\vspace{-\baselineskip} 
			\begin{enumerate} 
				\item Verdadeiro. 
				
				Podemos dividir em duas situações:
				
				1) Se A é singular e B também é singular:
				Para algum $x \neq 0$: $Bx = 0$.
				Então: $(AB)x = A(Bx) = 0$ e podemos concluir que AB também é singular.
				
				2) Se A é singular e B não é singular:
				Para algum $y \neq 0$: $Ay = 0$
				
				Então a partir de $Bz = y$, temos que $z \neq 0$ pois $y \neq 0$.
				Logo, $(AB)z = A(Bz) = Ay = 0$ e novamente podemos concluir que AB também é singular.
				
				\item Falso.
				
				Contra-exemplo: $A = \begin{bmatrix}
					0 & 1 \\
					1 & 0
				\end{bmatrix}$
			
				$\det{A} = 0 - 1 = -1$
				
				Mas seus pivôs são 1 e 1. Então o determinante de $A$ nem sempre é o produto de seus pivôs.
				
				\item Falso.
				
				Contra-exemplo: $A = \begin{bmatrix}
					1 & 0 \\
					0 & 0
				\end{bmatrix}$ e $B = \begin{bmatrix}
					0 & 0 \\
					0 & -1
				\end{bmatrix}$
				
				$A - B = \begin{bmatrix}
					1 & 0 \\
					0 & 1
				\end{bmatrix}$
				
				Sabendo que $\det(A) = 0$ e $\det(B) = 0$, então $\det(A) - \det(B) = 0 - 0 = 0$ mas ao calcular $\det(A - B)$ note que seu determinante é 1. Então $\det(A - B) \neq \det(A) - \det(B)$
				 
				\item Verdadeiro. Para \textbf{matrizes quadradas} que terão matrizes $AB$ e $BA$, suas determinantes são iguais.
				
				$\det(AB) = \det(A) \cdot \det(B) = \det(B) \cdot \det(A) = \det(BA)$
				
				Ps: $\det(A) \cdot \det(B) = \det(B) \cdot \det(A)$ são iguais por comutatividade.
			\end{enumerate}    
		\end{sol} 
		
		%%%%%%%%%%%%%%%%%%%%%%%%%%%%%%%%%%%%%%%%%%%%%%%%%%%%%%%%%
		%%%%%%%%%%%%%%%%%%%%%% Exercício 2 %%%%%%%%%%%%%%%%%%%%%%
		%%%%%%%%%%%%%%%%%%%%%%%%%%%%%%%%%%%%%%%%%%%%%%%%%%%%%%%%%
		
		\item Sejam $u$ e $v$ vetores ortonormais em $\bR^2$ e defina $A = uv^T$. Calcule $A^2$ para descobrir os autovalores de $A$. Verifique que o traço de $A$ é $\lambda_1 + \lambda_2$.
		
		\begin{sol} 
			$A = uv^T = \begin{bmatrix}
				u_1 \\
				u_2
			\end{bmatrix}\begin{bmatrix}
				v_1 & v_2
			\end{bmatrix} = \begin{bmatrix}
			u_1v_1 & u_1v_2 \\
			u_2v_1 & u_2v_2
			\end{bmatrix}$\\
		
			$A^2 = \begin{bmatrix}
				u_1v_1 & u_1v_2 \\
				u_2v_1 & u_2v_2
			\end{bmatrix}\begin{bmatrix}
				u_1v_1 & u_1v_2 \\
				u_2v_1 & u_2v_2
			\end{bmatrix} = \begin{bmatrix}
			(u_1v_1)^2 + (u_1v_2)(u_2v_1) & (u_1v_1)(u_1v_2) + (u_1v_2)(u_2v_2)\\
			(u_2v_1)(u_1v_1) + (u_2v_2)(u_2v_1) & (u_2v_1)(u_1v_2) + (u_2v_2)^2
			\end{bmatrix}$     
		
		= $\begin{bmatrix}
			(u_1v_1)^2 + u_1u_2v_1v_2 & u_1^2v_1v_2 + u_1u_2v_2^2\\
			v_1^2u_1u_2 + u_2^2v_1v_2 & u_1u_2v_1v_2 + (u_2v_2)^2
		\end{bmatrix} = \begin{bmatrix}
			(u_1v_1)(u_1v_1 + u_2v_2) & u_1v_2(u_1v_1 + u_2v_2)\\
			u_1v_2(u_1v_1 + u_2v_2) & (u_1v_1 + u_2v_2)(u_2v_2)
		\end{bmatrix}$
	
		Seja $u = \begin{bmatrix}
			u_1 \\
			u_2
		\end{bmatrix}$ e $v = \begin{bmatrix}
			v_1 \\
			v_2
		\end{bmatrix}$ vetores ortonormais, então: $u^Tv = 0 \Longleftrightarrow u_1v_1 + u_2v_2 = 0$, $u_1^2 + u_2^2 = 1$ e $v_1^2 + v_2^2 = 1$.  
	
		Então $A^2 = \begin{bmatrix}
			0 & 0\\
			0 & 0
		\end{bmatrix}$
	
		Os autovalores de $A^2$ são 0, e por isso os autovalores de $A$ são 0 também. Pois se autovalores de A são $\leftthreetimes$, autovalores de $A^2$ são $\leftthreetimes^2$. 
		\end{sol} 
		
		%%%%%%%%%%%%%%%%%%%%%%%%%%%%%%%%%%%%%%%%%%%%%%%%%%%%%%%%%
		%%%%%%%%%%%%%%%%%%%%%% Exercício 3 %%%%%%%%%%%%%%%%%%%%%%
		%%%%%%%%%%%%%%%%%%%%%%%%%%%%%%%%%%%%%%%%%%%%%%%%%%%%%%%%%
		
		\item A matriz $B$ tem autovalores 1 e 2, $C$ tem autovalores 3 e 4 e $D$ tem autovalores 5 e 7 (todas são matrizes $2 \times 2$). Ache os autovalores de $A$:
		$$A = \begin{bmatrix}
			B & C \\
			0 & D
		\end{bmatrix}.$$
		
		\begin{sol} 
			Sabendo que $B$, $C$ e $D$ são matrizes $2 \times 2$. Podemos supor que:\\
			$B = \begin{bmatrix}
				1 & 0 \\
				0 & 2
			\end{bmatrix}$ para que seus autovalores sejam 1 e 2.\\
			$C = \begin{bmatrix}
				3 & 0 \\
				0 & 4
			\end{bmatrix}$ para que seus autovalores sejam 3 e 4.\\
			$D = \begin{bmatrix}
				5 & 0 \\
				0 & 7
			\end{bmatrix}$ para que seus autovalores sejam 5 e 7.\\
		
		Então: $$A = \begin{bmatrix}
			1 & 0 & 3 & 0\\
			0 & 2 & 0 & 4\\
			0 & 0 & 5 & 0\\
			0 & 0 & 0 & 7
		\end{bmatrix}$$ 
	
	$$\det(A - \leftthreetimes I) = (1 - \leftthreetimes)(2 - \leftthreetimes)(5 - \leftthreetimes)(7 - \leftthreetimes)$$  
	
		Note que seus autovalores serão $\leftthreetimes_1 = 1$, $\leftthreetimes_2 = 2, \leftthreetimes_3 = 5$ e $\leftthreetimes_4 = 7$. Em outras palavras, os autovalores de A são os autovalores de B e D.
	
		\end{sol} 
		
		%%%%%%%%%%%%%%%%%%%%%%%%%%%%%%%%%%%%%%%%%%%%%%%%%%%%%%%%%
		%%%%%%%%%%%%%%%%%%%%%% Exercício 4 %%%%%%%%%%%%%%%%%%%%%%
		%%%%%%%%%%%%%%%%%%%%%%%%%%%%%%%%%%%%%%%%%%%%%%%%%%%%%%%%%
		
		\item Seja $D$ uma matriz $n \times n$ só com 1's em suas entradas. Procure a inversa da matriz $A = I + D$ dentre as matrizes $I + cD$ e ache o número $c$ correto.
		
		\begin{sol} 
		Supondo que $n = 2$:
		$$A = \begin{bmatrix}
				1 & 0 	\\
				0 & 1 	 \\
		\end{bmatrix} + \begin{bmatrix}
			1 & 1	 \\
			1 & 1 	 \\
		\end{bmatrix} = \begin{bmatrix}
			2 & 1 	\\
			1 & 2 	 \\
		\end{bmatrix}$$  
	
		$$A^{-1} = \begin{bmatrix}
			\frac{2}{3} & \frac{-1}{3} 	\\
			\frac{-1}{3} & \frac{2}{3} 	 \\
		\end{bmatrix}$$
	
		$$\begin{bmatrix}
			\frac{2}{3} & \frac{-1}{3} 	\\
			\frac{-1}{3} & \frac{2}{3} 	 \\
		\end{bmatrix} = \begin{bmatrix}
			1 & 0 	\\
			0 & 1 	 \\
		\end{bmatrix} + c\begin{bmatrix}
			1 & 1	 \\
			1 & 1 	 \\
		\end{bmatrix}$$
	
		$$c = \frac{-1}{3}$$
		
		Supondo que $n = 3$:
		
		$$A = \begin{bmatrix}
			1 & 0 & 0	\\
			0 & 1 & 0	 \\
			0 & 0 & 1
		\end{bmatrix} + \begin{bmatrix}
			1 & 1 & 1	 \\
			1 & 1 & 1	 \\
			1 & 1 & 1
		\end{bmatrix} = \begin{bmatrix}
			2 & 1 & 1	\\
			1 & 2 & 1	 \\
			1 & 1 & 1
		\end{bmatrix}$$  
		
		$$A^{-1} = \begin{bmatrix}
			\frac{3}{4} & \frac{-1}{4} & \frac{-1}{4}	\\
			\frac{-1}{4} & \frac{3}{4} & \frac{-1}{4}	 \\
			\frac{-1}{4} & \frac{-1}{4} & \frac{3}{4}
		\end{bmatrix}$$
	
		$$\begin{bmatrix}
			\frac{3}{4} & \frac{-1}{4} & \frac{-1}{4}	\\
			\frac{-1}{4} & \frac{3}{4} & \frac{-1}{4}	 \\
			\frac{-1}{4} & \frac{-1}{4} & \frac{3}{4}
		\end{bmatrix} = \begin{bmatrix}
			1 & 0 & 0	\\
			0 & 1 & 0	 \\
			0 & 0 & 1
		\end{bmatrix} + c\begin{bmatrix}
			1 & 1 & 1	 \\
			1 & 1 & 1	 \\
			1 & 1 & 1
		\end{bmatrix}$$
		
		$$c = \frac{-1}{4}$$
		
		
	Em outras palavras, note que A sempre será uma matriz de 1's com 2 na sua diagonal. Então, ao calcular a sua inversa, teremos:
	
	$$A^{-1} = \begin{bmatrix}
		\frac{n}{n+1} & \frac{-1}{n+1} & \frac{-1}{n+1}	& \cdots \\
		\frac{-1}{n+1} & \frac{n}{n+1} & \frac{-1}{n+1}	& \cdots \\
		\frac{-1}{n+1} & \frac{-1}{n+1} & \frac{n}{n+1}	& \cdots \\
		\vdots & \vdots & \vdots & \ddots 
		\end{bmatrix} = \begin{bmatrix}
		1 & 0 & 0 & \cdots \\
		0 & 1 & 0 & \cdots \\
		0 & 0 & 1 & \cdots \\
		\vdots & \vdots & \vdots & \ddots 
		\end{bmatrix} + c\begin{bmatrix}
		1 & 1 & 1 & \cdots	 \\
		1 & 1 & 1 & \cdots	 \\
		1 & 1 & 1 & \cdots   \\
		\vdots & \vdots & \vdots & \ddots 
		\end{bmatrix}$$

	Então $c = \frac{-1}{n+1}$.
		
			\end{sol} 
		
		%%%%%%%%%%%%%%%%%%%%%%%%%%%%%%%%%%%%%%%%%%%%%%%%%%%%%%%%%
		%%%%%%%%%%%%%%%%%%%%%% Exercício 5 %%%%%%%%%%%%%%%%%%%%%%
		%%%%%%%%%%%%%%%%%%%%%%%%%%%%%%%%%%%%%%%%%%%%%%%%%%%%%%%%%
		
		\item Vamos resolver uma EDO de segunda ordem usando o que aprendemos. Considere $y'' = 5y' + 4y$ com $y(0) = C_1$ e $y'(0) = C_2$. Defina $u_1 = y$ e $u_2 = y'$. Escreva $\bfu'(t) = A\bfu(t)$ e ache a solução da equação.
		
		\begin{sol} 
			$u_1 = y$\\
			$u_1' = y'$\\
			$u_1' = y' = 0u_1 + u_2$\\
			
			$u_2 = y'$\\
			$u_2' = y''$\\
			$y'' = 5y' + 4y$\\
			$u_2' = y'' = 5u_2 + 4u_1$\\
			
			
			$u =  \begin{bmatrix}
				u_1 \\
				u_2
			\end{bmatrix} $
		
		$u(0) = \begin{bmatrix}
			C1 \\
			C2
		\end{bmatrix}$
		
		$u_1(0) = y(0) = C1$\\
		$u_2(0) = y'(0) = C2$
		
		$u' = Au \Rightarrow \begin{bmatrix}
			u_1' \\
			u_2'
		\end{bmatrix} = \begin{bmatrix}
			0 & 1 \\
			4 & 5
		\end{bmatrix}\begin{bmatrix}
			u_1 \\
			u_2
		\end{bmatrix}$
	
		$A = \begin{bmatrix}
			0 & 1 \\
			4 & 5
		\end{bmatrix}$
	
		Achando os autovalores de A:
		
		$$\det(A - \leftthreetimes I) = \det \begin{bmatrix}
			-\leftthreetimes & 1 \\
			4 & 5 - \leftthreetimes
		\end{bmatrix} = (\leftthreetimes^2 - 5\leftthreetimes) - 4 = 0$$
	
		$\leftthreetimes^2 - 5\leftthreetimes - 4 = 0$\\
		
		$\bigtriangleup = 25 + 16 = 41$\\
		
		$\leftthreetimes = \frac{5 \pm \sqrt{41}}{2}$\\
		
		Achando os autovetores de A:
		
		$$ \begin{bmatrix}
			\frac{-5 - \sqrt{41}}{2} & 1 \\
			4 & \frac{5 - \sqrt{41}}{2}
		\end{bmatrix} \begin{bmatrix}
			x_1 \\
			x_2
		\end{bmatrix} = \begin{bmatrix}
			0 \\
			0
		\end{bmatrix}$$
	
		$$-(\frac{5 + \sqrt{41}}{2})x_1 + x_2 = 0$$
		$$4x_1 + (\frac{5 - \sqrt{41}}{2})x_2 = 0$$
		
		$v_1 = \begin{bmatrix}
			1 \\
			\frac{5 + \sqrt{41}}{2} 
		\end{bmatrix}$
	
		$$ \begin{bmatrix}
			\frac{-5 + \sqrt{41}}{2} & 1 \\
			4 & \frac{5 + \sqrt{41}}{2}
		\end{bmatrix} \begin{bmatrix}
			x_1 \\
			x_2
		\end{bmatrix} = \begin{bmatrix}
			0 \\
			0
		\end{bmatrix}$$
	
		$$-(\frac{5 - \sqrt{41}}{2})x_1 + x_2 = 0$$
		$$4x_1 + (\frac{5 + \sqrt{41}}{2})x_2 = 0$$
	
		$v_2 = \begin{bmatrix}
			1 \\
			\frac{5 - \sqrt{41}}{2} 
		\end{bmatrix}$
	
	
		A solução com as condições: $u(t) = B_1 v_1 e^{\leftthreetimes_1 t} + B_2 v_2 e^{\leftthreetimes_2 t}$
		
		$u(0) = B_1 v_1 + B_2 v_2 = \begin{bmatrix}
			C1 \\
			C2
		\end{bmatrix}$

		$u(0) = B_1 \begin{bmatrix}
			1 \\
			\frac{5 + \sqrt{41}}{2} 
		\end{bmatrix} + B_2 \begin{bmatrix}
			1 \\
			\frac{5 - \sqrt{41}}{2} 
		\end{bmatrix} = \begin{bmatrix}
			C1 \\
			C2
		\end{bmatrix}$

		$B_1 + B_2 = C_1$
		
		$B_1 \frac{5 + \sqrt{41}}{2} + B_2 \frac{5 - \sqrt{41}}{2}  = C_2$
		\end{sol} 
		
		%%%%%%%%%%%%%%%%%%%%%%%%%%%%%%%%%%%%%%%%%%%%%%%%%%%%%%%%%
		%%%%%%%%%%%%%%%%%%%%%% Exercício 6 %%%%%%%%%%%%%%%%%%%%%%
		%%%%%%%%%%%%%%%%%%%%%%%%%%%%%%%%%%%%%%%%%%%%%%%%%%%%%%%%%
		
		\item Se $A$ é simétrica e todos seus autovalores são iguais a $\lambda$. O que podemos dizer sobre $A$?
		
		\begin{sol} 
			  Se todos seus autovalores são iguais a $\lambda$, então A é uma matriz diagonal, cujo essa diagonal tem suas entradas iguais a $\lambda$. Pois então: $\det(A - \alpha I) = (\leftthreetimes - \alpha)^n = 0$ e $\alpha = \leftthreetimes$.
		\end{sol} 
		
		%%%%%%%%%%%%%%%%%%%%%%%%%%%%%%%%%%%%%%%%%%%%%%%%%%%%%%%%%
		%%%%%%%%%%%%%%%%%%%%%% Exercício 7 %%%%%%%%%%%%%%%%%%%%%%
		%%%%%%%%%%%%%%%%%%%%%%%%%%%%%%%%%%%%%%%%%%%%%%%%%%%%%%%%%
		
		\item Suponha que $C$ é positiva definida e que $A$ tenha as colunas LI. Mostre que $A^TCA$ é positiva definida.
		
		\begin{sol} 
		Em álgebra linear, uma matriz $A$ é positiva definida se $v^T A v > 0$ para todos $v$ em $R$.
		Então basta provar que $x^T(A^TCA)x > 0$:
		
		$x^T(A^TCA)x = (x^TA^T)C(Ax) = (Ax)^T C (Ax)$ 
			
		Visto que A tem colunas independentes, tem classificação completa e logo, é invertível. Para qualquer x, podemos encontrar um $y$ tal que $Ax = y$ tomando $x = A^{-1}y$.
		
		Então supondo que $Ax = y$, então: $x^T(A^TCA)x = y^T Cy$  
		
		$y^T Cy$ é positiva para qualquer $y$ pois C é positiva definida. Portanto, $A^TCA$ é positiva definida.
		\end{sol} 
		
		%%%%%%%%%%%%%%%%%%%%%%%%%%%%%%%%%%%%%%%%%%%%%%%%%%%%%%%%%
		%%%%%%%%%%%%%%%%%%%%%% Exercício 8 %%%%%%%%%%%%%%%%%%%%%%
		%%%%%%%%%%%%%%%%%%%%%%%%%%%%%%%%%%%%%%%%%%%%%%%%%%%%%%%%%
		
		\item Quais são os autovalores de $A$ se ela for similar a $A^{-1}$?
		
		\begin{sol} 
			Se $A$ for similar a $A^{-1}$ todos os autovalores de $A$ serão $1$ ou $-1$.   
			
			Isso acontece pois, visto que $A$ e $A^{-1}$ são similares, então elas terão os mesmos autovalores. Se um dos autovalores de A é $\leftthreetimes$, o autovalor de seu inverso será $\frac{1}{\leftthreetimes}$. Mas sabemos que eles são iguais, então:
			
			$$\leftthreetimes = \frac{1}{\leftthreetimes}$$
			$$\leftthreetimes^2 = 1$$
			$$\leftthreetimes = \pm 1$$
		\end{sol} 
		
		%%%%%%%%%%%%%%%%%%%%%%%%%%%%%%%%%%%%%%%%%%%%%%%%%%%%%%%%%
		%%%%%%%%%%%%%%%%%%%%%% Exercício 9 %%%%%%%%%%%%%%%%%%%%%%
		%%%%%%%%%%%%%%%%%%%%%%%%%%%%%%%%%%%%%%%%%%%%%%%%%%%%%%%%%
		
		\item Suponha que $A$ é quadrada, mostre que $\sigma_1 \geq |\lambda|$, para qualquer autovalor $\lambda$ de $A$, onde $\sigma_1$ é o primeiro valor singular de $A$.
		
		\begin{sol} 
			A norma de uma matriz é definida por $||A|| = max_{x \neq 0} \frac{||Ax||}{||x||}$  
			
		1. $||A|| \geq |\leftthreetimes|$, $\leftthreetimes$ é autovalor de A.
		
		Prova:
		
		Se $\leftthreetimes$ é autovalor de A e $v$ é autovetor de A, $Av = \leftthreetimes v$
		
		$||A|| = max_{x \neq 0} \frac{||Ax||}{||x||} \geq \frac{||Av||}{||v||} = \frac{||\leftthreetimes v||}{||v||} = \frac{|\leftthreetimes| ||v||}{||v||} = |\leftthreetimes|$
		
		$$||A|| \geq |\leftthreetimes|$$
		
		2.  $||AB|| \leq ||A|| ||B||$
		
		Prova:
		
		$||A|| = max_{x \neq 0} \frac{||Ax||}{||x||} \Rightarrow ||A|| \geq \frac{||Ax||}{||x||}$, para algum $x \neq 0$.
		
		$||Ax|| \leq ||A|| ||x||$
		
		Por outro lado,  $||A|| \geq \frac{||A(Bx)||}{||Bx||} = \frac{||ABx||}{||Bx||}$
		
		$||ABx|| \leq ||A|| ||Bx|| = ||A|| ||B|| ||x||$
		
		$\frac{||ABx||}{||x||} = ||A|| ||B||$, $\forall$ todo x.
		
		$$\Rightarrow ||AB|| \leqslant ||A||||B||$$
		
		3. $\sigma_1 \geq |\lambda|$
		
		Prova:
		
		$A = U \sum V^T \Rightarrow ||A|| = ||U \sum V^T||$
		
		$U$ e $V^T$ são ortogonais e tem norma 1.
		
		$$||U|| ||\sum|| ||V^T|| = ||\sum|| = \sigma_1$$
		
		$$||A|| = \sigma_1$$
		
		Conclusão:
		
		Sabendo que $||A|| \geq |\leftthreetimes|$ e $||A|| = \sigma_1$, então: 
		
		$$\sigma_1 \geq |\leftthreetimes|$$
		
		\end{sol} 
		
		%%%%%%%%%%%%%%%%%%%%%%%%%%%%%%%%%%%%%%%%%%%%%%%%%%%%%%%%%
		%%%%%%%%%%%%%%%%%%%%%% Exercício 10 %%%%%%%%%%%%%%%%%%%%%%
		%%%%%%%%%%%%%%%%%%%%%%%%%%%%%%%%%%%%%%%%%%%%%%%%%%%%%%%%%
		
		\item Ache a decomposição SVD da matriz
		$$A = \begin{bmatrix}
			1 & 0 & 1 & 0\\
			0 & 1 & 0 & 1 \\
		\end{bmatrix}.$$
		
		\begin{sol} 
			$A^T = \begin{bmatrix}
				1 & 0 \\
				0 & 1 \\
				1 & 0 \\
				0 & 1
			\end{bmatrix}$  
		
		$A^TA = \begin{bmatrix}
			1 & 0 \\
			0 & 1 \\
			1 & 0 \\
			0 & 1
		\end{bmatrix}\begin{bmatrix}
		1 & 0 & 1 & 0\\
		0 & 1 & 0 & 1 \\
		\end{bmatrix} = 
		\begin{bmatrix}
			1 & 0 & 1 & 0 \\
			0 & 1 & 0 & 1 \\
			1 & 0 & 1 & 0 \\
			0 & 1 & 0 & 1 
		\end{bmatrix}$
	
		Encontrando os autovalores de $A^TA$:
		
		$\det(A^TA - \leftthreetimes I) = (\leftthreetimes^2 - 2\leftthreetimes)^2 = (\leftthreetimes(\leftthreetimes - 2))^2 = 0$
		
		$\leftthreetimes_1 = 2$, $\leftthreetimes_2 = 2$, $\leftthreetimes_3 = 0$ e $\leftthreetimes_4 = 0$
		
		Encontrando os autovetores de $A^TA$:
		
		$v_1 = (1,0,1,1)$, $v_2 = (0,1,0,1)$, $v_4 = (-1,0,1,0)$, $v_4 = (0,-1,0,1)$
		
		Encontrando as raízes quadradas dos autovalores diferentes de zero $(\sigma i):$
		
		$\sigma_1 = \sqrt{2}$
			
		$\sigma_2 = \sqrt{2}$
		
		Então a matriz $\sum = \begin{bmatrix}
			\sqrt{2} & 0 & 0 & 0\\
			0 & \sqrt{2} & 0 & 0 \\
		\end{bmatrix}$
	
		As colunas da matriz V são os vetores normalizados (unitários), então a matriz $V = \begin{bmatrix}
			\frac{\sqrt{2}}{2} & 0 & \frac{-\sqrt{2}}{2} & 0 \\
			0 & \frac{\sqrt{2}}{2} & 0 & \frac{-\sqrt{2}}{2} \\
			\frac{\sqrt{2}}{2} & 0 & \frac{\sqrt{2}}{2} & 0 \\
			0 & \frac{\sqrt{2}}{2} & 0 & \frac{\sqrt{2}}{2} 
		\end{bmatrix}$
	
		Por último, $$u_1 = \frac{1}{\sigma_1} \cdot A \cdot v_1$$
		
		$$u_1 = \frac{1}{\sqrt{2}} \begin{bmatrix}
			1 & 0 & 1 & 0\\
			0 & 1 & 0 & 1 \\
		\end{bmatrix}\begin{bmatrix}
			\frac{\sqrt{2}}{2} \\
			0  \\
			\frac{\sqrt{2}}{2} \\
			0 
		\end{bmatrix} = \begin{bmatrix}
		1 \\
		0  
		\end{bmatrix}$$
	
		$$u_2 = \frac{1}{\sigma_2} \cdot A \cdot v_2$$
		
		$$u_2 = \frac{1}{\sqrt{2}} \begin{bmatrix}
			1 & 0 & 1 & 0\\
			0 & 1 & 0 & 1 \\
		\end{bmatrix}\begin{bmatrix}
			0 \\
			\frac{\sqrt{2}}{2}  \\
			0 \\
			\frac{\sqrt{2}}{2} 
		\end{bmatrix} = \begin{bmatrix}
			0 \\
			1  
		\end{bmatrix}$$
	
	$U = \begin{bmatrix}
		1 & 0 \\
		0 & 1  
	\end{bmatrix}$ 

		Em síntese, nossa decomposição SVD é dada por:
		
		$$A = \begin{bmatrix}
			1 & 0 \\
			0 & 1  
		\end{bmatrix}\begin{bmatrix}
		\sqrt{2} & 0 & 0 & 0\\
		0 & \sqrt{2} & 0 & 0 \\
		\end{bmatrix} \begin{bmatrix}
		\frac{\sqrt{2}}{2} & 0 & \frac{\sqrt{2}}{2} & 0 \\
		0 & \frac{\sqrt{2}}{2} & 0 & \frac{\sqrt{2}}{2} \\
		\frac{-\sqrt{2}}{2} & 0 & \frac{\sqrt{2}}{2} & 0 \\
		0 & \frac{-\sqrt{2}}{2} & 0 & \frac{\sqrt{2}}{2} 
	\end{bmatrix}$$
		
		\end{sol} 
	\end{enumerate}
	
	
	
	
	
	
	
	
	
	
	
	
	
	
	
\end{document} 