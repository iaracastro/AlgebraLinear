\documentclass[paper=a4]{article}
\makeatletter
\usepackage[utf8x]{inputenc}
\usepackage[brazil]{babel}
\usepackage[top=2cm,left=3cm,right=3cm,bottom=2cm]{geometry}
\usepackage{graphicx}
\usepackage{mathptmx}
\usepackage{url}
\usepackage[svgnames]{xcolor}
\usepackage{float} 
\textheight=700px
\frenchspacing
\usepackage{amsfonts,amsmath,amssymb,amsthm}
\usepackage{gensymb}
\usepackage{cancel}
\usepackage[hang]{footmisc}
%%%%%%%%%%%%%%%%%%%%%%%%%%%%%%%%%%%%%%%%%%%%%%%%%%%%%%%%%%%
\renewcommand{\@seccntformat}[1]{%
	\ifcsname prefix@#1\endcsname
	\csname prefix@#1\endcsname
	\else
	\csname the#1\endcsname\quad
	\fi}
%%%%%%%%%%%%%%%%%%%%%%%%%%%%%%%%%%%%%%%%%%%%%%%%%%%%%%%%%%%
	\newcommand\prefix@section{Questão \thesection: }
	\makeatother
%%%%%%%%%%%%%%%%%%%%%%%%%%%%%%%%%%%%%%%%%%%%%%%%%%%%%%%%%%%
\begin{document}
	\thispagestyle{empty}
	\begin{figure}
		\centering
		\includegraphics[width=0.6\linewidth]{C:/Users/iaram/OneDrive/Imagens/Prints/logo}
		\label{fig:logo}
	\end{figure}
	\vspace{7cm}
	\begin{center}	
		\textcolor{black}{{\Huge{\bf Respostas da Lista 1}}} \\	
	\end{center}
	\vspace{0,2cm}
	\begin{center}
		\textcolor{black}{{\Huge{\bf Álgebra Linear}}}
	\end{center}
	\vspace{0,8cm}	
	\begin{center}
		\textcolor{black}{{\LARGE{\bf Iara Cristina Mescua Castro}}} \\
	\end{center}
	\vspace{0,2cm}
	\begin{center}
		\textcolor{black}{{\Large{\bf Matrícula: 211704019}}}
	\end{center}
	\vspace{0,5cm}
	%\begin{center}
	%	\textcolor{black}{{\LARGE{\bf Juliana Carrica Barcellos Miranda}}} \\
	%\end{center}
	%\vspace{0,7cm}
	\begin{center}
		\textcolor{black}{\Large{\bf 09/08/2021}}
	\end{center}
	
	\newpage
	\thispagestyle{empty}
	\tableofcontents
	%%%%%%%%%%%%%%%%%%%%%%%%%%%%%%%%%%%%%%%%%%%%%%%%%%%%%%%%%%%%%%%%%%%%%%%%%%%%%%%%%%%%%%%
	\newpage
	\thispagestyle{empty}
	%\listoffigures
	\newpage
	\setcounter{page}{1}
	\pagenumbering{arabic}
	
	%%QUESTÃO 1
	%%%%%%%%%%%%%%%%%%%%%%%%%%%%%%%%%%%%%%%%%%%%%%%%%%%%%%%%%%%%%%%%%%%%%%%%%%%%%%%%%%%%%%%

	\section{\vspace{0,1cm}\begin{flushleft}
			{\large Quais condições para y1, y2 e y3 fazem com que os pontos (0, y1), (1, y2) e (2, y3) caiam numa reta?}
	\end{flushleft}}
    {\large 
    Para três pontos estarem na mesma reta, as coordenadas dos pontos devem cumprir a seguinte condição:}
    \begin{center}{\Large$
       \det	\begin{vmatrix}
    		x_1 & y_1 & 1\\
    		x_2 & y_2 & 1 \\
    		x_3 & y_3 & 1
    	\end{vmatrix} = 0$} \end{center}
    {\large Substituindo os valores:}
    \begin{center}{\Large$
    \det\begin{vmatrix}
    		0 & y_1 & 1\\
    		1 & y_2 & 1 \\
    		2 & y_3 & 1
    	\end{vmatrix} = 0$} 
    \end{center}
    {\large Resolvendo a determinante:
    	\begin{center}
    	$
    	\begin{vmatrix}
    		0 & y_1 & 1\\
    		1 & y_2 & 1 \\
    		2 & y_3 & 1
    	\end{vmatrix}
    	\begin{vmatrix}
    		0 & y_1\\
    		1 & y_2\\
    		2 & y_3
    	\end{vmatrix} = 0$
    \end{center}}
	{\large $0 + 2y_1 + y_3 - (2y_2 + 0 + y_1)$\\
	\newline
	$y_1 - 2y_2 + y_3 = 0$}\\
	\newline
	{\large Então para os pontos $(0,y_1)$, $(1,y_2)$ e $(2,y_3)$ estarem na mesma reta,\\
	\newline
	eles devem satisfazer a equação: $y_1 - 2y_2 + y_3 = 0$}
	\newpage
	
	%%QUESTÃO 2
	%%%%%%%%%%%%%%%%%%%%%%%%%%%%%%%%%%%%%%%%%%%%%%%%%%%%%%%%%
	
	\section{\vspace{0,1cm}\begin{flushleft}
			{\large Se (a, b) é um múltiplo de (c, d) e são todos não-zeros, mostre que (a, c) é um múltiplo de (b, d).  O que isso nos diz sobre a matriz}
	\end{flushleft}}

	\begin{center}{\Large$
	A =	\begin{bmatrix}
			a& b\\
			c& d
		\end{bmatrix}$} \end{center}
	
	{\large
	Se (a, b) é um múltiplo de (c, d), então:\\	
	\begin{center}
			$(a, b) = x(c, d)$\\
			$a = xc$\\
			$b = xd$\\
			$c = a/x$\\
			$d = b/x$
	\end{center}
	Então se (a, c) é múltiplo de (b, d):\\
	\begin{center}
			$(a, c) = y(b, d)$\\
	\end{center}
	Precisamos obter:
	\begin{center}
			$a = yb$\\
			$c = yd$
	\end{center}
	Substituindo $a$ e $b$ em $(a, c) = y(b, d)$:\\
	\begin{center}
			$(xc, c) = y(xd, d)$\\
			$\cancel{x}c = y\cancel{x}d$\\
			$c = yd$ 
	\end{center}
	Substituindo $c$ e $d$ em $(a, c) = y(b, d)$:
	\begin{center}
			$(a, a/x) = y(b, b/x)$\\
			\vspace{0.1cm}
			$\frac{a}{\cancel{x}} = y\frac{b}{\cancel{x}}$\\
			\vspace{0.1cm}
			$a = yb$ 
	\end{center}
	Se substituirmos $a$, $b$, $c$ e $d$ na matriz e calcularmos sua determinante:
	\begin{center}{\Large$
			A =	\begin{bmatrix}
				xc & xd\\
				a/x & b/x
	\end{bmatrix}$} \end{center}
	Calculando a determinante: 
	$xc . \frac{b}{x} - (\frac{a}{x} . xd) =  cb - ad$
	Calculando a determinante da matriz original:
	\begin{center}{\Large$
		A =	\begin{bmatrix}
			a& b\\
			c& d
		\end{bmatrix}$} \end{center}
	\begin{flushleft}
		$a . d - c b$\\
		Com isso, é possível concluir que:\\
		cb - ad = ad - cb\\
	 	2cb = 2ad\\
	 	cb = ad
	\end{flushleft}
 	E assim, podemos concluir que a determinante dessa matriz A é igual a 0, é nulo.
	\newpage
	
	%%QUESTÃO 3
	%%%%%%%%%%%%%%%%%%%%%%%%%%%%%%%%%%%%%%%%%%%%%%%%%%%%%%%%%
	
	\section{\vspace{0,1cm}\begin{flushleft}
			{\large Se w e v são vetores unitários, calcule os produtos internos de\vspace{0,2cm}
			\\(a) $v$ e $−v$;
			\\(b) $v + w$ e $v − w$;
			\\(c) $v − 2w$ e $v + 2w$}\\
			\end{flushleft}}
	\begin{center}
		\textbf{(a)} $<v, -v> = ||v|| · ||-v|| · \cos180$\\
		\vspace{0,1cm}
		$||v|| = 1$, $||-v|| = 1$ e $\cos180 = -1$, logo:\\
		\vspace{0,1cm}
		$= (1) · (1) ·(-1)$ \\
		\vspace{0,1cm}
		$= -1$
	\end{center}
	\begin{center}
		\textbf{(b)} $<v + w, v - w> = <v,v> - <v,w> + <w,v> - <w,w>$\\
		\vspace{0,1cm}
		$<v,v> = 1$, $<v,w> = <w,v>$ e $<w,w> = 1$, logo:\\
		\vspace{0,1cm}
		$= 1 + 0 - 1$ \\
		\vspace{0,1cm}
		$= 0$
	\end{center}
	\begin{center}
		\textbf{(c)} $<v - 2w, v + 2w> = <v,v> + <v,2w> - <2w,v> - <2w,2w>$\\
		\vspace{0,1cm}
		$<v,v> = 1$, $<v,2w> = <2w,v>$ e $<2w,2w> = 4<w,w> = 4$, logo:\\
		\vspace{0,1cm}
		$= 1 + 0 - 4$ \\
		\vspace{0,1cm}
		$= -3$
	\end{center}
	\newpage
	
	%%QUESTÃO 4
	%%%%%%%%%%%%%%%%%%%%%%%%%%%%%%%%%%%%%%%%%%%%%%%%%%%%%%%%%
	
	\section{\vspace{0,1cm}\begin{flushleft}
			{\large Se $||v|| = 5$ e $||w|| = 3$, quais são o menor e maior valores possíveis para $||v − w||$? E para $v · w$?}
	\end{flushleft}} 
	{\large Primeiro vamos provar que: $||v| - |w|| ≤ ||v - w||$ \newline
	\vspace{0,1cm}
	Através da desigualdade triangular, $||a||+||b|| ≥ ||a + b||$,
	substituímos a = v - w  e  b = w, \newline formando:
	\begin{center}
		$||v - w|| + ||w|| ≥ ||v - w + w||$ \\
		\vspace{0,1cm}
    	$||v - w|| ≥ ||v| - |w||$
	\end{center}
	Por isso, quando v ≥ w ≥ 0
	\vspace{0,1cm}
	\begin{center}
		$||v - w|| ≥ ||v| - |w||$\\
		\vspace{0,1cm}
		$||v - w|| ≥ |5 - 3|$\\
		\vspace{0,1cm}
		$||v - w|| ≥ |2|$\\
		\vspace{0,1cm}
		$||v - w|| ≥ 2$\\
	\end{center}
	\textbf{2} é o \textbf{menor} valor possível de $|v - w|$\\
	\newline
	E quando v ≥ 0 ≥ w , ou seja, v = 5 e w = -3
		\begin{center}
		$||v - w|| = ||5 - (-3)||$\\
		\vspace{0,1cm}
		$||v - w|| ≤ ||8|| $\\
		\vspace{0,1cm}
		$||v - w|| ≤ 8 $
	\end{center}
	\textbf{8} é o \textbf{maior} valor possível de $|v - w|$\\
	\newline
	\vspace{0,3cm}
	Para $v \cdot w$, o menor valor possível seria negativo, ou seja, um dos valores será negativo e o outro positivo.\\
	$v > 0$ e $w < 0$, ou seja, v = 5 e w = -3 \\
	\vspace{0,1cm}
	Esses valores satisfazem as condições, pois\\
	\vspace{0,1cm}
	$||v|| = 5$ e $||5|| = 5$\\
	\vspace{0,1cm}
	$||w|| = 3$ e $||-3|| = 3$\\
	\vspace{0,1cm}
	Assim, o menor valor para $v \cdot w$ é igual a 5 . (-3) = \textbf{-15}\\
	\newline
	\vspace{0,3cm}
	E o maior valor seria positivo, ou seja, os dois valores serão positivos (ou os dois serão negativos).\\
	$v > 0$ e $w > 0$, ou seja, v = 5 e w = 3 \\
	\vspace{0,1cm}
	Esses valores satisfazem as condições, pois\\
	\vspace{0,1cm}
	$||v|| = 5$ e $||5|| = 5$\\
	\vspace{0,1cm}
	$||w|| = 3$ e $||3|| = 3$\\
	\vspace{0,1cm}
	Assim, o maior valor para $v \cdot w$ é igual a 5 . 3 = \textbf{15}
	\newpage
	
	%%QUESTÃO 5
	%%%%%%%%%%%%%%%%%%%%%%%%%%%%%%%%%%%%%%%%%%%%%%%%%%%%%%%%%
	
	\section{\vspace{0,1cm}\begin{flushleft}
			{\large Considere o desenho dos vetores w e v abaixo. Hachure as regiões definidas pelas combinações lineares
				cv + dw considerando as seguintes restrições: c + d = 1 (não necessariamente positivos), c, d ∈ [0, 1] e
				c, d ≥ 0 (note que são três regiões distintas).}
	\end{flushleft}}
	\begin{figure}[H]
		\centering
		\includegraphics[width=0.7\linewidth]{"5 item 1"}
		\caption{}
		\label{fig:5-item-1}
	\end{figure}
	\vspace{3cm}
	\begin{figure}[H]
		\centering
		\includegraphics[width=0.6\linewidth]{"5 item 2"}
		\caption{Questão 5, item 2}
		\label{fig:5-item-2}
	\end{figure}
	\newpage
	\begin{figure}[H]
		\centering
		\includegraphics[width=0.6\linewidth]{"5 item 3"}
		\caption{Questão 5, item 3}
		\label{fig:5-item-3}
	\end{figure}
	
	\newpage
	
	%%QUESTÃO 6
	%%%%%%%%%%%%%%%%%%%%%%%%%%%%%%%%%%%%%%%%%%%%%%%%%%%%%%%%%
	
	\section{\vspace{0,1cm}\begin{flushleft}
		{\large E possível que três vetores em $\mathbb{R}^2$ tenham\vspace{0,2cm} $u · v < 0$, $v · w < 0$ e $u · w < 0?$
		\\Argumente.}
	\end{flushleft}}
	Sim, é possível. Primeiro, sem perda de generalidade, vamos supor que u, v, w são vetores unitários.
	\newline
	$ u . v = cos \measuredangle(u,v) < 0$\\
	\vspace{0.1cm}
	Logo, $\measuredangle(u,v) \in (\frac{2\pi}{3}, \pi)$\\
	\vspace{0.25cm}
	$ v . w = cos (v,w) < 0$\\
	\vspace{0.1cm}
	Logo, $\measuredangle(v,w) \in (\frac{2\pi}{3}, \pi)$\\
	\vspace{0.25cm}
	$ u . w = cos (u,w) < 0$\\
	\vspace{0.1cm}
	Logo, $\measuredangle(u,w) \in (\frac{2\pi}{3}, \pi)$\\
	\newline
	No plano cartesiano, podemos criar a seguinte situação:
	\begin{figure}[H]
		\centering
		\includegraphics[width=0.7\linewidth]{"6"}
		\caption{Questão 6}
		\label{fig:6}
	\end{figure}
	E assim, forma-se 3 vetores (u, v, w) que estão em $\mathbb{R}^2$ e satisfazem essas condições.
	
	\newpage
	
	%%QUESTÃO 7
	%%%%%%%%%%%%%%%%%%%%%%%%%%%%%%%%%%%%%%%%%%%%%%%%%%%%%%%%%
	
	\section{\vspace{0,1cm}\begin{flushleft}
		{\large Sejam x, y, z satisfazendo x + y + z = 0. \\
		Calcule o ângulo entre os vetores (x, y, z) e (z, x, y)}
	\end{flushleft}}
	Primeiro vamos chamar: \\
	u = (x, y, z)\\
	v = (z, x, y)\\
	z = -x -y ou z = -(x+y)
	\begin{center}
		$cos\measuredangle(u,v) = \frac{(x, y, z) . (x, y, z)}{||u|| . ||v||}$\\
		\vspace{0.3cm}
		$cos\measuredangle(u,v) = \frac{xz + xy + zy}{||u|| . ||v||}$\\
		\vspace{0.3cm}
		$cos\measuredangle(u,v) = \frac{x(z+y) + zy}{||u|| . ||v||}$\\
		\vspace{0.3cm}
		$cos\measuredangle(u,v) = \frac{x(-x) + (-x-y)y}{||u|| . ||v||}$\\
		\vspace{0.3cm}
		$cos\measuredangle(u,v) = \frac{-x^2 -xy + y^2}{||u|| . ||v||}$\\
		\vspace{0.3cm}
		$cos\measuredangle(u,v) = \frac{-(x^2 + xy + y^2)}{||u|| . ||v||}$\\
		\vspace{0.3cm}
	\end{center}
	Calculando as normas:\\
	\newline
	$||u|| = \sqrt{x² + y² + x² + 2xy + y²} = \sqrt{2x² + 2xy + 2y²}$ \\
	$||v|| = \sqrt{z² + x² + y²} = \sqrt{2x² + 2xy + 2y²}$\\
	$||u|| . ||v|| = 2x² + 2xy + 2y² = 2(x^2 + 2xy + 2y^2)$\\
	\newline
	Substituindo:
	\begin{center}
		$cos\measuredangle(u,v) = \frac{-(x^2 + xy + y^2)}{2(x^2 + 2xy + 2y^2)}$\\
		\vspace{0.3cm}
		$cos\measuredangle(u,v) = \frac{-1}{2}$\\
		\vspace{0.3cm}
		$\measuredangle(u,v) = 240^{\circ}$\\
	\end{center}

	\newpage
	
	%%QUESTÃO 8
	%%%%%%%%%%%%%%%%%%%%%%%%%%%%%%%%%%%%%%%%%%%%%%%%%%%%%%%%%
	
	\section{\vspace{0,1cm}\begin{flushleft}
		{\large Resolva o sistema linear abaixo:
		\begin{center}{\Large$
			\begin{bmatrix}
			    1 & 0 & 0\\
				1 & 1 & 0 \\
				1 & 1 & 1
			\end{bmatrix}
		    \begin{bmatrix}
		     	x_1\\
		     	x_2\\
		     	x_3
		    \end{bmatrix} =
	        \begin{bmatrix}
	    	    b_1\\
	    	    b_2\\
	    	    b_3
	        \end{bmatrix}
		$} \end{center}  
		Escreva a solução x como uma matriz A vezes o vetor b.}
	\end{flushleft}}
	Temos o sistema com as equações lineares:\\
	\vspace{0.3cm}
	$
	\left\{
	\begin{array}{c}
		1x_1 + 0x_2 + 0x_3 = b_1\\
		1x_{1} + 1x_{2} + 0x_3 = b_2\\
		1x_1 + 1x_2 + 1x_3 = b_3
	\end{array}
	\right.
	$\\
	Por isso,
	\begin{center}
		$x_1 = b_1$\\
		\vspace{0.5cm}
		$x_1 + x_2 = b_2$\\
		\vspace{0.3cm}
		$b_1 + x_2 = b_2$\\
		\vspace{0.3cm}
		$x_2 = b_2 - b_1$\\
		\vspace{0.5cm}
		$x_1 + x_2 + x_3 = b_3$\\
		\vspace{0.3cm}
		$\cancel{b_1} + b_2 \cancel{-b_1} + x_3 = b_3$\\
		\vspace{0.3cm}
		$x_3 = b_3 - b_2$\\
	\end{center}
	Em outras palavras,\\
	\newline
	$
	\left\{
	\begin{array}{c}
		1b_1 + 0b_2 + 0b_3 = x_1\\
		-1b_{1} + 1b_{2} + 0b_3 = b_2\\
		0b_1 + -1b_2 + 1b_3 = x_3
	\end{array}
	\right.
	$\\
	\newline
	Assim, a solução de x, como uma matriz A vezes o vetor b fica:
	\begin{center}{\Large$
		\begin{bmatrix}
			1 & 0 & 0\\
			-1 & 1 & 0 \\
			0 & -1 & 1
		\end{bmatrix}
		\begin{bmatrix}
			b_1\\
			b_2\\
			b_3
		\end{bmatrix} =
		\begin{bmatrix}
			x_1\\
			x_2\\
			x_3
		\end{bmatrix}
	$} \end{center}  
	\newpage
	
	%%QUESTÃO 9
	%%%%%%%%%%%%%%%%%%%%%%%%%%%%%%%%%%%%%%%%%%%%%%%%%%%%%%%%%
	\section{\vspace{0,1cm}\begin{flushleft}
		{\large Repita o problema acima para a matriz:} \\
		\begin{center}{\Large$
			A =	\begin{bmatrix}
				-1 & 1 & 0\\
				0 & -1 & 1 \\
				0 & 0 & -1
		\end{bmatrix}$} \end{center}	
	\end{flushleft}}
	Repetindo o processo, temos o sistema com as equações lineares:\\
	\vspace{0.3cm}
	$
	\left\{
	\begin{array}{c}
		-1x_1 + 1x_2 + 0x_3 = b_1\\
		0x_{1} - 1x_{2} + 1x_3 = b_2\\
		0x_1 + 0x_2 + -1x_3 = b_3
	\end{array}
	\right.
	$\\
	Por isso,
	\begin{center}
		$x_3 = -b_3$\\
		\vspace{0.5cm}
		$0_1 - 1x_2 + 1x_3= b_2$\\
		\vspace{0.3cm}
		$0_1 - 1x_2 - b_3 = b_2$\\
		\vspace{0.3cm}
		$x_2 = -(b_2 + b_3)$\\
		\vspace{0.5cm}
		$-x_1 + 1x_2 + 0x_3 = b_1$\\
		\vspace{0.3cm}
		$-x_1 -(b_2 + b_3) + 0(-b_3) = b_1$\\
		\vspace{0.3cm}
		$x_1 = -b_2 -b_3 - b_1$\\
		\vspace{0.3cm}
		$x_1 = -(b_2 + b_3)$\\
	\end{center}
	Em outras palavras,\\
	\newline
	$
	\left\{
	\begin{array}{c}
		-1b_1 - 1b_2 - 1b_3 = x_1\\
		0b_{1} - 1b_{2} - 1b_3 = x_2\\
		0b_1 + 0b_2 - 1b_3 = x_3
	\end{array}
	\right.
	$\\
	\newline
	Assim, a solução de x, como uma matriz A vezes o vetor b fica:
	\begin{center}{\Large$
		\begin{bmatrix}
			-1 & -1 & -1\\
			0 & -1 & -1 \\
			0 & 0 & -1
		\end{bmatrix}
		\begin{bmatrix}
			b_1\\
			b_2\\
			b_3
		\end{bmatrix} =
		\begin{bmatrix}
			x_1\\
			x_2\\
			x_3
		\end{bmatrix}
		$} \end{center}  
	\newpage
	
	%%QUESTÃO 10
	%%%%%%%%%%%%%%%%%%%%%%%%%%%%%%%%%%%%%%%%%%%%%%%%%%%%%%%%%
	
	\section{\vspace{0,1cm}\begin{flushleft}
		{\large Considere a equação de recorrência $−x_{i+1} + 2_{xi} −x_{i−1} = i$ para $i = 1, 2, 3, 4$ com $x_0 = x_5 = 0$. Escreva essas equações em notação matricial $Ax = b$ e ache x.}
	\end{flushleft}}
	Equação
	\begin{center}
		$-x_{i+1} + 2x_i - x_{i-1} = i$
	\end{center}
	Para i = 1:
	\begin{center}
		$-x_2 +  2x_{1} - x_{0} = 1$\\
		$2x_1 - x_2 + 0 + 0 = 1$
	\end{center}
	Para i = 2:
	\begin{center}
		$-x_3 +  2x_{2} - x_{1} = 2$\\
		$-x_{1} + 2x_{2} - x_3 + 0 = 2$
	\end{center}
	Para i = 3:
	\begin{center}
		$-x_4 + 2x_{3} - x_{2} = 3$\\
		$0 -x_2 + 2x_3 - x_4 = 3$
	\end{center}
	Para i = 4:
	\begin{center}
		$-x_5 + 2x_{4} - x_{3} = 4$\\
		$0 + 0 - x_3{3} + 2x_4 = 4$
	\end{center}
	Juntando todas as equações em um sistema, obtemos:\newline
	\newline
	$
	\left\{
	\begin{array}{c}
	2x_1 - x_2 + 0 + 0 = 1\\
	-x_{1} + 2x_{2} - x_3 + 0 = 2\\
	0 -x_2 + 2x_3 - x_4 = 3\\
	0 + 0 - x_3 + 2x_4 = 4
	\end{array}
	\right.
	$
	\newline
	Escrevendo essas equações em notação matricial:
	$\begin{bmatrix}
		2 & -1 & 0 & 0 \\
		-1 & 2 & -1 & 0 \\
		0 & -1 & 2 & -1 \\
		0 & 0 & -1 & 2
	\end{bmatrix} 
	\begin{bmatrix}
		x_1\\
		x_2\\
		x_3\\
		x_4
	\end{bmatrix} =
	\begin{bmatrix}
		1\\
		2\\
		3\\
		4
	\end{bmatrix}
		$\\
	Somando:
	\begin{center}
	$1x_1 + 0x_2 + 0x_3 + 1x_4 = 10$\\
	$x_1 = 10 - x_4$
	\end{center}
	Substituindo em $L_1$:\\
	\begin{center}
		$2(10 - x_4) -x_2 = 1$\\
		$20 - 2x_4 -x_2 = 1$\\
		$2x_4 + x_2 = 19$\\
		$x_2 = 19 -2x_4$\\
	\end{center}
	Isolando $x_3$ em $L_4$:\\
	\begin{center}
		$2x_4 = 4 + 1x_3$\\
		$x_3 = 2x_4 - 4$\\
	\end{center}
	\newpage
	Substituindo tudo em $L_2$:
	\begin{center}
		$-(10 - x_4) + 2(19 -2x_4) - (2x_4 -4) = 2$\\
		$x_4 - 10 + + 38 - 4x_4 - 2x_4 + 4 = 2$\\
		$x_4 - 10 + + 38 - 4x_4 - 2x_4 + 4 = 2$\\
		$-5x_4 - 10 + 38 + 4 = 2$\\
		$-5x_4 - 10 + 38 + 4 = 2$\\
		$-5x_4 = -30$\\
		$x_4 = 6$\\
	\end{center}
	Substituindo nas equações:
	\begin{center}
		$x_3 = 2(6) - 4$\\
		$x_3 = 8$\\
		\vspace{0.2cm}
		$x_2 = 19 -2(6)$\\
		$x_2 = 7$\\
		\vspace{0.2cm}
		$x_1 = 4$
	\end{center}
	Resultado: $X = 
		\begin{bmatrix}
		4\\
		7\\
		8\\
		6
	\end{bmatrix}$
	\newpage
	
	%%%%%%%%%%%%%%%%%%%%%%%%%%%%%%%%%%%%%%%%%%%%%%%%%%%%%%%%%
	
	\section{\vspace{0,1cm}\begin{flushleft}
		{\large (Bônus) Use o seguinte código em numpy para gerar um vetor aleatório:
	v = numpy.random.normal(size=[3,1]) em $\mathbb{R}^3$. Fazendo $u = v/||v||$ criamos então um vetor unitário aleatório. Crie 30 outros vetores unitários aleatórios $u_j$ (use numpy.random.normal(size=[3,30])). Calcule a média dos produtos internos $|u · u_j|$ e compare com o valor exato $\frac{1}{\pi} \int_0^{\pi}|\cos\theta|d\theta = \frac{2}{\pi}$}
	\end{flushleft}}

	%%%%%%%%%%%%%%%%%%%%%%%%%%%%%%%%%%%%%%%%%%%%%%%%%%%%%%%%%
	\end{document}