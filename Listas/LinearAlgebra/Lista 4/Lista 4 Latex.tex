\documentclass[leqno]{article}
\usepackage[utf8x]{inputenc}
\usepackage[brazil]{babel} % \usepackage[latin1]{inputenc}
\usepackage{a4wide}
\setlength{\oddsidemargin}{-0.2in}
% % \setlength{\oddsidemargin}{0.2in}
\setlength{\evensidemargin}{-0.2in}
% % \setlength{\evensidemargin}{0.5in}
% % \setlength{\textwidth}{5.5in}
\setlength{\textwidth}{6.5in}
\setlength{\topmargin}{-1.2in}
\setlength{\textheight}{10in}
\usepackage[]{amsfonts} \usepackage[]{amsmath}
\usepackage[]{amssymb} \usepackage[]{latexsym}
\usepackage{graphicx,color} \usepackage{amsthm}
\usepackage{mathrsfs} \usepackage{url}
\usepackage{cancel} \usepackage{enumerate}
\usepackage{xifthen} \usepackage{tikz}
\usepackage{mathtools} 
\usepackage{hyperref}
\hypersetup{
	colorlinks=true,
	urlcolor=blue
}
\usetikzlibrary{automata,arrows,positioning,calc}
% \input{../preamble}

\newtheorem{teo}{Teorema}[section] 

\numberwithin{equation}{section}

\setlength{\parindent}{12 pt}

\begin{document}
	
	% \newtheorem{teo}{Teorema}[section] \newtheorem*{teo*}{Teorema}
	% \newtheorem{prop}[teo]{Proposição}
	\newtheorem*{prop*}{Proposição}
	\newtheorem{lema}[teo]{Lemma} \newtheorem*{lema*}{Lema}
	% \newtheorem{cor}[teo]{Corolário}
	\newtheorem*{cor*}{Corolário}
	
	\theoremstyle{definition}
	\newtheorem{defi}[teo]{Definição} \newtheorem*{defi*}{Definição}
	\newtheorem{exem}[teo]{Exemplo} \newtheorem*{exem*}{Exemplo}
	\newtheorem{obs}[teo]{Observação} \newtheorem*{obs*}{Observação}
	\newtheorem*{hipo}{Hipóteses}
	\newtheorem*{nota}{Notação}
	
	\newcommand{\ds}{\displaystyle} \newcommand{\nl}{\newline}
	\newcommand{\eps}{\varepsilon} \newcommand{\ssty}{\scriptstyle}
	\newcommand{\bE}{\mathbb{E}}
	\newcommand{\cB}{\mathcal{B}}
	\newcommand{\cF}{\mathcal{F}}
	\newcommand{\cA}{\mathcal{A}}
	\newcommand{\cM}{\mathcal{M}}
	\newcommand{\cD}{\mathcal{D}}
	\newcommand{\cN}{\mathcal{N}}
	\newcommand{\cL}{\mathcal{L}}
	\newcommand{\cLN}{\mathcal{LN}}
	\newcommand{\bP}{\mathbb{P}}
	\newcommand{\bQ}{\mathbb{Q}}
	\newcommand{\bN}{\mathbb{N}}
	\newcommand{\R}{\mathbb{R}}
	\newcommand{\bZ}{\mathbb{Z}}
	
	\DeclarePairedDelimiter{\dotprod}{\langle}{\rangle} 
	\newcommand{\defeq}{\vcentcolon=}
	\newcommand{\bfw}{\mathbf{w}}
	\newcommand{\bfv}{\mathbf{v}}
	\newcommand{\bfu}{\mathbf{u}}
	
	\newcommand{\bvecc}[2]{%
		\begin{bmatrix} #1 \\ #2  \end{bmatrix}
	}
	\newcommand{\bveccc}[3]{%
		\begin{bmatrix} #1 \\ #2 \\ #3  \end{bmatrix}
	}
	
	\newenvironment{sol}
	{
		\vspace{4mm}
		\noindent\textbf{Resolução:}
		\strut\newline
		\smallskip
		\hspace{-3.5mm}
	}
	{} 
	
	\title{Álgebra Linear - Lista de Exercícios 4}
	
	\author{Iara Cristina Mescua Castro}
	
	\date{}
	
	\maketitle
	
	\begin{enumerate}
		
		\item Sejam $S$ e $T$ dois subespaços de um espaço vetorial $V$.
		
		\begin{enumerate}
			
			\item Defina $S + T = \{s + t \ ; \ s \in S \mbox{ e } t \in T\}$. Mostre que $S + T$ é um subespaço vetorial.
			
			\begin{sol} 
				\newline					
				Visto que 0 $\in S + T$ , pois $0 \in S$ e $0 \in T$ e $0 + 0 = 0$.\\
				Vamos supor que a $\in$ S + T e b $\in$ S + T.\\
				Dessa forma: 
				\begin{center}
				a = s′ + t′\\
				b = s′′ + t′′\\
				\end{center} 
				Onde: 
				\begin{center}
				s′, s′′ $\in$ S\\
				t′, t′′$\in$ T\\
				\end{center}
				Então:
				\begin{center}
				a + b = s′ + t′ + s′′ + t′′\\
				Por comutatividade de adição:\\
				= (s′ + s′′) + (t′ + t′′)\\
				\end{center} 
				Lembrando que: s′ + s′′$\in$ S e t ′ + t′′$\in$ T uma vez que S e T são subespaços.\\
				Assim, a + b $\in$ S + T.\\
				
				Por último, suponha a $\in$ S + T.\\
				Portanto, a = s′ + t′.\\
				Seja c algum escalar, então:
				\begin{center}
				c $\times$ a = c (s′ + t′) = cs′ + ct′\\ 
				c $\times$ s′$\in$ S e c $\times$ t′ $\in$ T
				\end{center}
				Uma vez que S e T são subespaços. Assim, c $\times$ a $\in$ S + T.
				
				Isso prova que S + T é um subespaço.
				\newline
			\end{sol} 
			
			\item Defina $S \cup T = \{x \ ; \ x \in S \mbox{ ou } x \in T\}$. Argumente que $S \cup T$ não é necessariamente um subespaço vetorial.
			
			\begin{sol} 
				\newline
				Isso acontece pois todos os espaços vetoriais e, portanto, também os subespaços, devem ser fechados sob adição e multiplicação escalar. Mas na união dos subespaços $S$ e $T$, existem novas combinações de vetores, como $v_1 + v_2$ onde $v_1 \in S$ e $v_2 \in T$.\\
				\newline
				Contraexemplo:\\
				Supondo que S é o eixo x e T seja o eixo y, onde ambos são subespaços de $\mathbb{R}^2$.\\
				Sua união contém os pontos (0,1) e (1,0), cuja soma (1,1), não está na união.
				\newline
			\end{sol} 
			
			\item Se $S$ e $T$ são retas no $\R^3$, o que é $S + T$ e $S \cup T$?
			
			\begin{sol} 
				$S + T$ é um plano em $\R^3$\\
				$S \cup T$ são todos os pontos contidos na reta S e na reta T.
			\end{sol} 
		\end{enumerate}
		
		\item Como o núcleo $N(C)$ é relacionado aos núcleos $N(A)$ e $N(B)$, onde $C = \begin{bmatrix}A \\ B \end{bmatrix}$?
		
		\begin{sol} 
			\newline
			A partir de uma matriz $A_{m \times n}$ e $B_{k \times n}$, onde $C = \begin{bmatrix}A \\ B \end{bmatrix}$\\
			$N(A) = \{x \in \mathbb{R}^n \mbox{:} A \times x = 0\}$\\
			$N(B) = \{x \in \mathbb{R}^n \mbox{:} B \times x = 0\}$\\
			$N(C) = \{x \in \mathbb{R}^n \mbox{:} \begin{bmatrix} A \\ B \end{bmatrix}
			\times x = 0\}$   \\
			\newline
			--------------------------------------------------------------------------------------\\
			${\displaystyle A\mathbf {x} =\mathbf {0} \;\;\Leftrightarrow \;\;{\begin{alignedat}{7}a_{11}x_{1}&&\;+\;&&a_{12}x_{2}&&\;+\;\cdots \;+\;&&a_{1n}x_{n}&&\;=\;&&&0\\a_{21}x_{1}&&\;+\;&&a_{22}x_{2}&&\;+\;\cdots \;+\;&&a_{2n}x_{n}&&\;=\;&&&0\\&&&&&&&&&&\vdots \ \;&&&\\a_{m1}x_{1}&&\;+\;&&a_{m2}x_{2}&&\;+\;\cdots \;+\;&&a_{mn}x_{n}&&\;=\;&&&0{\text{.}}\\\end{alignedat}}}$\\
			--------------------------------------------------------------------------------------\\
			${\displaystyle B\mathbf {x} =\mathbf {0} \;\;\Leftrightarrow \;\;{\begin{alignedat}{7}b_{11}x_{1}&&\;+\;&&b_{12}x_{2}&&\;+\;\cdots \;+\;&&b_{1n}x_{n}&&\;=\;&&&0\\b_{21}x_{1}&&\;+\;&&b_{22}x_{2}&&\;+\;\cdots \;+\;&&b_{2n}x_{n}&&\;=\;&&&0\\&&&&&&&&&&\vdots \ \;&&&\\b_{k1}x_{1}&&\;+\;&&b_{k2}x_{2}&&\;+\;\cdots \;+\;&&b_{kn}x_{n}&&\;=\;&&&0{\text{.}}\\\end{alignedat}}}$\\
			--------------------------------------------------------------------------------------\\
			${\displaystyle \begin{bmatrix} A \\ B \end{bmatrix} \mathbf {x} =\mathbf {0} \;\;\Leftrightarrow
			\;\;{\begin{alignedat}{7}A_{11}x_{1}&&\;+\;&&A_{12}x_{2}&&\;+\;\cdots \;+\;&&A_{1n}x_{n}&&\;=\;&&&0\\A_{21}x_{1}&&\;+\;&&A_{22}x_{2}&&\;+\;\cdots \;+\;&&A_{2n}x_{n}&&\;=\;&&&0\\&&&&&&&&&&\vdots \ \;&&&\\A_{m1}x_{1}&&\;+\;&&A_{m2}x_{2}&&\;+\;\cdots \;+\;&&a_{mn}x_{n}&&\;=\;&&&0{\text{.}}\\&&&&&&&&&&\vdots \ \;&&&\\b_{(k+m)1}x_{1}&&\;+\;&&b_{(k+m)2}x_{2}&&\;+\;\cdots \;+\;&&b_{(k+m)n}x_{n}&&\;=\;&&&0{\text{.}}\\\end{alignedat}}}$ \\
			--------------------------------------------------------------------------------------\\
			\newline
			Se $C \cdot x = 0$, $A \cdot x = 0$ e $B \cdot x = 0$. Assim, $N(C) = N(A) | N(B)$ = interseção.
		\end{sol} 
		
		\item Considere a matriz
		$$A = \begin{bmatrix} 
			1 & 5 & 7 & 9\\
			0 & 4 & 1 & 7 \\
			2 & -2 & 11 & -3
		\end{bmatrix}.$$
		
		\begin{enumerate}
			
			\item Ache a sua forma escalonada reduzida.
			
			\begin{sol} 
				$$A = \begin{bmatrix} 
					1 & 5 & 7 & 9\\
					0 & 4 & 1 & 7 \\
					2 & -2 & 11 & -3
				\end{bmatrix}$$
			$L_{3} \longleftrightarrow L_{3} - 2 \cdot L_{1}$\\
				$$\begin{bmatrix} 
					1 & 5 & 7 & 9\\
					0 & 4 & 1 & 7 \\
					0 & -12 & -3 & -21
				\end{bmatrix}$$
			$L_{2} \longleftrightarrow \frac{L_{2}}{4}$\\
				$$\begin{bmatrix} 
					1 & 5 & 7 & 9\\
					0 & 1 & \frac{1}{4} & \frac{7}{4} \\
					0 & -12 & -3 & -21
				\end{bmatrix}$$
			$L_{1} \longleftrightarrow L_{1} - 5 \cdot L_{2}$\\
			$$\begin{bmatrix} 
				1 & 0 & \frac{23}{4} & \frac{1}{4}\\
				0 & 1 & \frac{1}{4} & \frac{7}{4} \\
				0 & -12 & -3 & -21
			\end{bmatrix}$$
			$L_{4} \longleftrightarrow L_{4} + 12 \cdot L_{2}$\\
			$$\begin{bmatrix} 
				1 & 0 & \frac{23}{4} & \frac{1}{4}\\
				0 & 1 & \frac{1}{4} & \frac{7}{4} \\
				0 & 0 & 0 & 0
			\end{bmatrix}$$\\
			$$R = \begin{bmatrix} 
				I_{2 \times 2} & F_{2 \times 2}\\
				0_{1 \times 2} & 0_{1 \times 2} 
			\end{bmatrix}$$\\
			Onde, $$F = \begin{bmatrix} 
				\frac{23}{4} & \frac{1}{4}\\
				\frac{1}{4} & \frac{7}{4} 
			\end{bmatrix}$$
			\end{sol} 
			
			\item Qual é o posto dessa matriz?
			
			\begin{sol} 
				O posto é 2. É o número de linhas não-nulas da matriz na forma escalonada.  
			\end{sol} 
			
			\item Ache uma solução especial para a equação $Ax = 0$.
			
			\begin{sol} 
				$Ax = 0 \longleftrightarrow Rx = 0$\\
				Então, $$\begin{bmatrix} 
					1 & 0 & \frac{23}{4} & \frac{1}{4}\\
					0 & 1 & \frac{1}{4} & \frac{7}{4} \\
					0 & 0 & 0 & 0
				\end{bmatrix}\begin{bmatrix} 
					x_1 \\
					x_2 \\
					x_3 \\
					x_4 \end{bmatrix} = 
					\begin{bmatrix} 
					0 \\
					0 \\
					0 \end{bmatrix}$$ \\
				$
				\left\{
				\begin {array}{cl}
				1 \cdot x_1 + 0 \cdot x_2 + x_3 \cdot \frac{23}{4} + x_4 \cdot \frac{1}{4} = 0\\
				0 \cdot x_1 + 1 \cdot x_2 + x_3 \cdot \frac{1}{4} + x_4 \cdot \frac{7}{4} = 0\\
				0 \cdot x_1 + 0 \cdot x_2 + 0 \cdot x_3 + 0 \cdot x_4 = 0\\
				\end{array}
				\right.$\\
				\newline
				$x_3 = 1$ e $x_4 = 0$\\
				\newline
				$
				\left\{
				\begin {array}{cl}
				1 \cdot x_1 + 0 \cdot x_2 + 1 \cdot \frac{23}{4} + 0 \cdot \frac{1}{4} = 0\\
				0 \cdot x_1 + 1 \cdot x_2 + 1 \cdot \frac{1}{4} + 0 \cdot \frac{7}{4} = 0\\
				0 \cdot x_1 + 0 \cdot x_2 + 0 \cdot 1 + 0 \cdot 0 = 0\\
				\end{array}
				\right.$\\
				\newline
				$x_1 = \frac{-23}{4}$\\
				\newline
				$x_2 = \frac{-1}{4}$
				\newline
				\newline
				$x \Rightarrow \begin{bmatrix} 
				\frac{-23}{4} \\
				\frac{-1}{4} \\
				1 \\
				0 \end{bmatrix}$
			\end{sol} 
		\end{enumerate}
		
		\item Ache a matrizes $A_1$ e $A_2$ (não triviais) tais que posto$(A_1B) = 1$ e posto$(A_2B) = 0$ para $B = \begin{bmatrix}1 & 1 \\ 1 & 1 \end{bmatrix}$.
		
		\begin{sol} 
			$A_1 = 
			\begin{bmatrix}
			1 & 1 \\ 1 & -1 
			\end{bmatrix}$\\
			\newline
			Pois, 
			$\begin{bmatrix}
				1 & 1 \\
				1 & -1 
			\end{bmatrix}
			\begin{bmatrix}
				1 & 1 \\
				1 & 1 
			\end{bmatrix} = 
			\begin{bmatrix}
				1 \cdot 1 + 1 \cdot 1 & 1 \cdot 1 + 1 \cdot 1\\
				1 \cdot 1 + (-1) \cdot 1 & 1 \cdot 1 + (-1) \cdot 1  
			\end{bmatrix} =
			\begin{bmatrix}
				2 & 2\\
				0 & 0  
			\end{bmatrix}$\\
			\newline
			Cujo, posto é 1 pois há 1 linha não-nulas nessa matriz.\\
			\newline
			$A_2 =  
			\begin{bmatrix}
			1 & -1 \\ 1 & -1 
			\end{bmatrix}$\\
			\newline
			Pois, 
			$
			\begin{bmatrix}
			1 & -1 \\
			1 & -1 
			\end{bmatrix}
			\begin{bmatrix}
			1 & 1 \\
			1 & 1 
			\end{bmatrix} = 
			\begin{bmatrix}
			1 \cdot 1 + (-1) \cdot 1 & 1 \cdot 1 + (-1) \cdot 1\\
			1 \cdot 1 + (-1) \cdot 1 & 1 \cdot 1 + (-1) \cdot 1  
			\end{bmatrix} =
			\begin{bmatrix}
				0 & 0\\
				0 & 0  
			\end{bmatrix}
			$\\
			\newline
			Cujo, posto é zero pois não há linhas não-nulas nessa matriz.
			\newline
		\end{sol} 
		
		\item Verdadeiro ou Falso:
		
		\begin{enumerate}
			
			\item O espaço das matrizes simétricas ($A^T = A$) é subespaço.
			
			\begin{sol} 
				Verdadeiro.\\
				\newline
				Para mostrar que um subconjunto S de um espaço vetorial V é um subespaço, precisamos atender que:\\
				1- O vetor zero em V está em S.\\
				2- Para quaisquer dois vetores $u, v \in S$ , temos $u + v \in S$\\
				3- Para qualquer escalar c e qualquer vetor $u \in S$, temos $c \cdot u \in S$\\
				\newline		
				Vamos considerar que o conjunto S consiste em todas as matrizes simétricas n × n.\\
				Então, vamos provar que S é um subespaço de V.\\
				\newline
				O vetor zero O em V é a matriz zero n × n e é simétrico. Assim, o vetor zero $O \in S$ e a condição 1 é atendidos.\\
				\newline
				Para verificar a segunda condição, considere quaisquer $A, B \in S$, ou seja, A e B são matrizes simétricas.
				Para mostrar que $A + B \in S$, precisamos verificar se a matriz A + B é simétrica.\\
				\newline
				Nós temos:
				$(A + B) T = A^T + B^T = A + B$\\
				Uma vez que A, B são simétricos. Assim, A + B também é simétrico, e $A + B \in S$.\\
				\newline
				Por último, para verificar a condição 3, deixe $A \in S$ e $r \in R$. Mostramos que $r \cdot A \in S$ ou seja, mostramos que $r \cdot A$ é simétrico.\\
				\newline
				Nós temos:
				$(r \cdot A) \cdot T = r \cdot A^T = r \cdot A$
				\newline
				Já que A é simétrico. Assim, $r \cdot A$ é simétrico e, portanto, $r \cdot A \in S$
				Assim, a condição 3 é atendida.\\
				\newline
				Pelos critérios de subespaço, o subconjunto S é um subespaço do espaço vetorial V.\\
				\newline
			\end{sol} 
			
			\item O espaço das matrizes anti-simétricas ($-A^T = A$) é um subespaço.
			
			\begin{sol} 
				Verdadeiro.\\
				\newline
				Agora, considerando que o conjunto S consiste em todas as matrizes anti-simétricas n × n.\\
				E da mesma forma, vamos verificar as 3 condições para provar que S é um subespaço de V.\\
				1- O vetor zero em V está em S.\\
				2- Para quaisquer dois vetores $u, v \in S$ , temos $u + v \in S$\\
				3- Para qualquer escalar c e qualquer vetor $u \in S$, temos $c \cdot u \in S$\\	
				\newline
				O vetor zero O em V é a matriz n × n, e é assimétrico porque: $O^T = O = −O$\\
				\newline
				Para a segunda condição, considere quaisquer $A, B \in S$, ou seja, A e B são matrizes anti-simétricas.\\
				Sabendo que: $A^T= −A$ e $B^T = −B$\\
				Para mostrar que $A + B \in S$, equivale a mostrar que a matriz A + B é assimétrica.\\
				Nós temos:
				$(A + B)^T = A^T + B^T = -A + (-B) = -(A + B)$\\
				\newline
				Por último, consideramos qualquer $A \in S$ e $r \in R$
				Mostramos que $r \cdot A$ é assimétrico e, por isso, $r \cdot A \in S$.\\
				\newline
				Usando o fato de que A é assimétrico (AT = −A), temos:\\
				$(r \cdot A) T = r \cdot A^T = r \cdot (−A) = -r \cdot A$
				\newline
			\end{sol} 
			
			\item O espaço das matrizes não-simétricas ($A^T \neq A$) é um subespaço.
			
			\begin{sol} 
				Falso.\\
				\newline
				Contra-exemplo:
				Temos as matrizes $A_1 = 
				\begin{bmatrix}
					1 & 0\\
					0 & 0\\ 
				\end{bmatrix}$ e $A_2 = 
				\begin{bmatrix}
					0 & 0\\
					0 & 1\\ 
				\end{bmatrix}$\\
			Que são matrizes não simétricas, mas sua soma é igual a:\\
			$A_1 + A_2 = \begin{bmatrix}
				1 & 0\\
				0 & 1\\ 
			\end{bmatrix}$
			Que é igual à matriz identidade, que \textbf{não} é uma matriz não-simétrica, e consequentemente, não pertencendo à S.\\
			Assim, acaba não cumprindo a segunda condição.
			\end{sol} 
			
		\end{enumerate}
		
		\item Se $A$ é $4\times 4$ e inversível, descreva todos os vetores no núcleo da matriz $B = \begin{bmatrix}A & A \end{bmatrix}$ (que é $4\times 8$).
		
		\begin{sol} 
			Supondo um vetor 8-dimensional $v = \left(
			\begin{array}{c}
				x\\
				y\\
			\end{array}
			\right)$
			
			onde x e y são dois vetores quadridimensionais. Então v está no
			núcleo de B se e somente se $B \cdot v = 0$. Mas, 
			$B \cdot v = \begin{pmatrix}
				A & A 
			\end{pmatrix}$
			$\left(
			\begin{array}{c}
				x\\
				y\\
			\end{array}
			\right) = A \cdot x + A \cdot y = A \cdot (x + y)$\\
			Então $B \cdot v = 0$ apenas se $A \cdot (x + y) = 0$\\
			Sabendo que A é invertível, $x + y = 0$ ou seja, $y = -x$
			Sendo assim, o núcleo de B é composto por todos os vetores na forma $\left(
			\begin{array}{c}
				x\\
				-x\\
			\end{array}
			\right)$ se x é um vetor quadridimensional.
		\end{sol} 
		
		\item Mostre por contra-exemplos que as seguintes afirmações são falsas em geral:
		
		\begin{enumerate}
			
			\item $A$ e $A^T$ tem os mesmo núcleos.
			
			\begin{sol} 
				$A = \begin{bmatrix}
					2 & 1\\
					-4 & -2\\ 
				\end{bmatrix}$ 
			Para resolver $A \cdot x = 0$, começamos reduzindo A:\\
			\newline
			$L_2 \longleftrightarrow L_2 + 2L_1$\\
			\newline
			$\begin{bmatrix}
				2 & 1\\
				0 & 0\\ 
			\end{bmatrix}$\\
			\newline
			Então: 	$\begin{bmatrix}
				2 & 1\\
				0 & 0\\ 
			\end{bmatrix}$$\begin{bmatrix}
			x_1 \\
			x_2
			\end{bmatrix} = \begin{bmatrix}
			0 \\
			0
			\end{bmatrix}$\\
			\newline
			$x_2$ pode ser tomado como a variável livre, e a primeira linha nos dá:\\
			$2x_1 + x_2 = 0$ ou $x_1 = \frac{-1}{2}x_2$
			\newline
			Então qualquer vetor na forma:$ \begin{bmatrix}
				\frac{-1}{2}x \\
				x
			\end{bmatrix}$ satisfaz $A \cdot x = 0$\\
			\newline
			$A^T = \begin{bmatrix}
				2 & -4\\
				1 & -2\\ 
			\end{bmatrix}$\\
			Para resolver $A^T \cdot x = 0$, começamos reduzindo A:\\
			\newline
			$L_2 \longleftrightarrow L_2 - \frac{1}{2}L_1$\\
			\newline
			$\begin{bmatrix}
				2 & -4\\
				0 & 0\\ 
			\end{bmatrix}$\\
			$x_2$ pode ser tomado como a variável livre, e a primeira linha nos dá:\\
			$2x_1 - 4x_2 = 0$ ou $x_1 = 2\cdot x_2$
			\newline
			Então qualquer vetor na forma:$ \begin{bmatrix}
				2x \\
				x
			\end{bmatrix}$ satisfaz $A^T \cdot x = 0$\\
			\newline
			Dessa forma, os núcleos de $A$ e $A^T$ não são iguais.
			\newline
			\end{sol} 
			
			\item $A$ e $A^T$ tem as mesmas variáveis livres.
			
			\begin{sol} 
				$A = \begin{bmatrix}
					2 & 1 & 3\\
					5 & 3 & 0\\ 
				\end{bmatrix}$ 
				Então,\\
				$\begin{bmatrix}
					2 & 1 & 3\\
					5 & 3 & 0\\ 
				\end{bmatrix} \begin{bmatrix}
				x_1\\
				x_2\\
				x_3
			\end{bmatrix} =  \begin{bmatrix}
			b_1\\
			b_2\\
			b_3
			\end{bmatrix}$\\
			\newline
			Essa matriz tem uma variável livre.\\
			\newline
				$A^T = \begin{bmatrix}
				2 & 5\\
				1 & 3\\ 
				3 & 0
			\end{bmatrix}
			\begin{bmatrix}
				x_1\\
				x_2
			\end{bmatrix} =  \begin{bmatrix}
				c_1\\
				c_2\\
				c_3
			\end{bmatrix}$\\
			\newline
			Essa matriz não tem nenhuma variável livre.\\
			\newline
			\end{sol} 
			
			\item Se $R$ é a forma escalonada de $A$, então $R^T$ é a forma escalonada de $A^T$.
			
			\begin{sol} 
				$A = \begin{bmatrix} 
					1 & 4\\
					1 & 3
				\end{bmatrix}$
			Escalonando:\\
			$L_2 \longleftrightarrow L_2 - L_1$\\
			\newline
			$\begin{bmatrix} 
				1 & 4\\
				0 & -1
			\end{bmatrix}$\\
			\newline
			$L_2 \longleftrightarrow -L_2$\\
			\newline
			$\begin{bmatrix} 
				1 & 4\\
				0 & 1
			\end{bmatrix} = R$\\
			\newline
			$R^T = \begin{bmatrix} 
				1 & 0\\
				4 & 1
			\end{bmatrix} $\\
			\newline
			Agora a transposta de A:\\
			\newline
			$A^T = \begin{bmatrix} 
				1 & 1\\
				4 & 3
			\end{bmatrix}$
			Escalonando:\\
			\newline
			$L_2 \longleftrightarrow L_2 - 4L_1$\\
			\newline
			$\begin{bmatrix} 
				1 & 1\\
				0 & -1
			\end{bmatrix}$\\
			\newline
			$L_2 \longleftrightarrow -L_2$\\
			\newline
			$\begin{bmatrix} 
				1 & 1\\
				0 & 1
			\end{bmatrix} \neq R^T$
			\end{sol} 
		\end{enumerate}
		
		\item Construa uma matriz cujo espaço coluna contenha $(1,1,5)$ e $(0,3,1)$ e cujo núcleo contenha $(1,1,2)$.
		
		\begin{sol} 
			A matriz será:$$\begin{bmatrix} 
				1 & 0 & y_1\\
				1 & 3 & y_2\\
				5 & 1 & y_3
			\end{bmatrix} $$ 
		Sabendo que o núcleo contém $(1,1,2)$, formamos:\\
		\newline
		$$\begin{bmatrix} 
			1 & 0 & y_1\\
			1 & 3 & y_2\\
			5 & 1 & y_3
		\end{bmatrix}\begin{bmatrix} 
		1\\
		1\\
		2
		\end{bmatrix} = \begin{bmatrix} 
		0\\
		0\\
		0
		\end{bmatrix}$$
		\newline
		$
		\left\{
		\begin {array}{cl}
		1 \cdot 1 + 0 \cdot 1 + y_1 \cdot 2 = 1 + 2y_1 = 0\\
		1 \cdot 1 + 3 \cdot 1 + y_2 \cdot 2 = 4 + 2y_2 = 0\\
		5 \cdot 1 + 1 \cdot 1 + y_3 \cdot 2 = 6 + 2y_3 = 0\\
		\end{array}
		\right.$\\
		\newline
		Então:\\
		$y_1 = \frac{-1}{2}$\\
		$y_2 = -2$\\
		$y_3 = -3$\\
		\newline
		Substituindo os valores, a matriz será: $$\begin{bmatrix} 
			1 & 0 & \frac{-1}{2}\\
			1 & 3 & -2\\
			5 & 1 & -3
		\end{bmatrix} $$\\
		\end{sol} 
		
		\item Construa uma matriz cujo núcleo contenha todos os múltiplos de $(4,3,2,1)$.
		
		\begin{sol} 
			A matriz:  $$A = \begin{bmatrix} 
				1 & 0 & 0 & -4\\
				0 & 1 & 0 & -3\\
				0 & 0 & 1 & -2
			\end{bmatrix} $$\\
		Encontrando o núcleo de A é:
		$$\begin{bmatrix} 
			1 & 0 & 0 & -4\\
			0 & 1 & 0 & -3\\
			0 & 0 & 1 & -2
		\end{bmatrix} \begin{bmatrix} 
		x_1 \\
		x_2 \\
		x_3 \\
		x_4 \end{bmatrix} =
		\begin{bmatrix} 
			0 \\
			0 \\
			0\end{bmatrix}$$\\
		\newline
		$
		\left\{
		\begin {array}{cl}
		1 \cdot x_1 + 0 \cdot x_2 + 0 \cdot x_3  - 4 \cdot x_4 \\
		0 \cdot x_1 + 1 \cdot x_2 + 0 \cdot x_3 - 3 \cdot x_4 \\
		0 \cdot x_1 + 0 \cdot x_2 + 1 \cdot x_3 - 2 \cdot x_4 \\
		\end{array}
		\right.$\\
		\newline
		$
		\left\{
		\begin {array}{cl}
		x_1 = 4 \cdot x_4\\
		x_2 = 3 \cdot x_4\\
		x_3 = 2 \cdot x_4\\
		\end{array}
		\right.$\\
		Então o núcleo de A $\rightarrow \begin{bmatrix} 
			4x \\
			3x \\
			2x \\
			1x \end{bmatrix}$, que contém múltiplos de $(4,3,2,1)$.
		\end{sol} 
		
		\item (\textit{Bônus}) Dado um espaço vetorial real \( V \), definimos o conjunto
		\begin{equation*}
			V^{ * } \defeq \left\{ f : V \to \R \mid f \text{ é linear} \right\}
			.\end{equation*}
		Ou seja, \( V^{ * } \) é o conjunto de todas as funções lineares entre \( V \) e \( \R \).
		Relembramos que uma função \( f : E \to F \), onde \( E \) e \( F \) são espaços vetoriais, é dita \textit{linear} se para todos \( \bfv, \bfw \in E \) e \( \alpha \in \R \) temos \( f ( \bfv + \bfw ) = f ( \bfv ) + f ( \bfw ) \) e \( f ( \alpha \bfv ) = \alpha f ( \bfv ) \).
		Chamamos \( V^{ * } \) de \textit{espaço dual} de \( V \).
		\begin{enumerate}
			\item Mostre que \( V^{ * } \) é um espaço vetorial.
			
			\begin{sol} 
				% escreva sua solução aqui.  
			\end{sol} 
			\item Agora, seja \( V = \R^{ n } \).
			Mostre que existe uma bijeção \( \varphi : V^{ * } \to V \) tal que , para toda \( f \in V^{ * } \) e para todo \( \bfv \in V \), tenhamos
			\begin{equation*}
				f(\bfv) = \dotprod{\varphi(f), \bfv}
				.\end{equation*} 
			\textit{Dica}: Utilize a dimensão finita de \( \R^{ n } \) para expandir \( \bfv \) como uma combinação linear dos vetores da base canônica e aplique a linearidade de \( f \).
			
			\begin{sol} 
				% escreva sua solução aqui.  
			\end{sol} 	
		\end{enumerate}
		Em dimensão infinita, esse resultado é conhecido como \href{https://en.wikipedia.org/wiki/Riesz_representation_theorem}{Teorema da Representação de Riesz}.
	\end{enumerate}
	
\end{document}