\documentclass[leqno]{article}

\usepackage[brazil]{babel} \usepackage[utf8]{inputenc}
\usepackage{a4wide}
\setlength{\oddsidemargin}{-0.2in}
% % \setlength{\oddsidemargin}{0.2in}
\setlength{\evensidemargin}{-0.2in}
% % \setlength{\evensidemargin}{0.5in}
% % \setlength{\textwidth}{5.5in}
\setlength{\textwidth}{6.5in}
\setlength{\topmargin}{-1.2in}
\setlength{\textheight}{10in}
\usepackage[]{amsfonts} \usepackage[]{amsmath}
\usepackage[]{amssymb} \usepackage[]{latexsym}
\usepackage{graphicx,color} \usepackage{amsthm}
\usepackage{mathrsfs} \usepackage{url}
\usepackage{cancel} \usepackage{enumerate} 
\usepackage{enumitem} 
\usepackage{xifthen} \usepackage{tikz}
\usetikzlibrary{automata,arrows,positioning,calc}

% \numberwithin{equation}{section}

\setlength{\parindent}{12 pt}

\begin{document}
	
	\newtheorem{teo}{Teorema} \newtheorem*{teo*}{Teorema}
	\newtheorem{prop}[teo]{Proposição} \newtheorem*{prop*}{Proposição}
	\newtheorem{lema}[teo]{Lemma} \newtheorem*{lema*}{Lema}
	\newtheorem{cor}[teo]{Corolário} \newtheorem*{cor*}{Corolário}
	
	\theoremstyle{definition}
	\newtheorem{defi}[teo]{Definição} \newtheorem*{defi*}{Definição}
	\newtheorem{exem}[teo]{Exemplo} \newtheorem*{exem*}{Exemplo}
	\newtheorem{obs}[teo]{Observação} \newtheorem*{obs*}{Observação}
	\newtheorem*{hipo}{Hipóteses}
	\newtheorem*{nota}{Notação}
	
	\newcommand{\ds}{\displaystyle} \newcommand{\nl}{\newline}
	\newcommand{\eps}{\varepsilon} \newcommand{\ssty}{\scriptstyle}
	\newcommand{\bE}{\mathbb{E}}
	\newcommand{\cB}{\mathcal{B}}
	\newcommand{\cF}{\mathcal{F}}
	\newcommand{\cA}{\mathcal{A}}
	\newcommand{\cM}{\mathcal{M}}
	\newcommand{\cD}{\mathcal{D}}
	\newcommand{\cN}{\mathcal{N}}
	\newcommand{\cL}{\mathcal{L}}
	\newcommand{\cLN}{\mathcal{LN}}
	\newcommand{\bP}{\mathbb{P}}
	\newcommand{\bQ}{\mathbb{Q}}
	\newcommand{\bN}{\mathbb{N}}
	\newcommand{\bR}{\mathbb{R}}
	\newcommand{\bZ}{\mathbb{Z}}
	
	\newcommand{\bfw}{\mathbf{w}}
	\newcommand{\bfv}{\mathbf{v}}
	\newcommand{\bfu}{\mathbf{u}}
	\newcommand{\bfx}{\mathbf{x}}
	\newcommand{\bfb}{\mathbf{b}}
	
	\newcommand{\bvecc}[2]{%
		\begin{bmatrix} #1 \\ #2  \end{bmatrix}
	}
	\newcommand{\bveccc}[3]{%
		\begin{bmatrix} #1 \\ #2 \\ #3  \end{bmatrix}
	}
	
	\newenvironment{sol}
	{
		\vspace{4mm}
		\noindent\textbf{Resolução:}
		\strut\newline
		\smallskip
		\hspace{-3.5mm}
	}
	
	% Objetos que aparecem *após* o ambiente. 
	% (você pode, por exemplo, modificar, 
	% ou remover, a barra horizontal} 
	%{\noindent\rule{4cm}{.1mm}}
	
	\title{Álgebra Linear - Lista de Exercícios 7}
	
	\author{Iara Cristina Mescua Castro}
	
	\date{\today}
	
	\maketitle
	
	\begin{enumerate}
		
		\item Se $AB = 0$, as colunas de $B$ estão em qual espaço fundamental de $A$? E as linhas de $A$ estão em qual espaço fundamental de $B$? É possível que $A$ e $B$ sejam $3 \times 3$ e com posto 2?
		
		\begin{sol} 
			Já que $AB = 0$, então:\\
			$$\begin{bmatrix}
				L_1 & \cdots & \cdots \\
				L_2 & \cdots & \cdots \\
				\vdots & \vdots & \vdots \\
				L_n & \cdots & \cdots \\
			\end{bmatrix}\begin{bmatrix}
				C_1 & C_2 & \cdots & C_p \\
				\vdots & \vdots & \vdots & \vdots \\
				\vdots & \vdots & \vdots & \vdots
			\end{bmatrix} = \begin{bmatrix}
			L_1C_1 & L_1C_2 & \cdots & L_1C_p \\
			L_2C_1 & L_2C_2 & \cdots & L_2C_p \\
			\vdots & \vdots & \vdots & \vdots \\
			L_nC_1 & L_nC_2 & \cdots & L_nC_p \\
		\end{bmatrix}= \begin{bmatrix}
		0 & 0 & \cdots & 0 \\
		0 & 0 & \cdots & 0 \\
		\vdots & \vdots & \vdots & \vdots \\
		0 & 0 & \cdots & 0 \\
		\end{bmatrix}$$
	
		Com isso, se pegarmos a primeira coluna da matriz resultante:
		$\begin{bmatrix}
			L_1C_1 \\
			L_2C_1 \\
			\vdots \\
			L_nC_1 \\
		\end{bmatrix}$\\
		É o mesmo que $A C_1$, e é igual a 0. Sendo assim, $C_1$ faz parte do núcleo de A. Ao analisar todas as colunas, teremos $A C_2$, $A C_3$, até $A C_p$ e já que todas são iguais a 0, podemos concluir que as colunas de B fazem parte de N(A).
		
		Analogamente, se calcularmos a transposta de $AB = 0$, que é $B^T A^T = 0$, e repetindo o mesmo processo, teremos que as colunas de B (que agora são as linhas), vezes as linhas de A (que agora são colunas), são iguais a 0. Então ao ver as colunas da matriz resultante, podemos concluir que as linhas de A fazem parte de $N(B^T)$. Representando essa situação:\\
		$$\begin{bmatrix}
			C_1 & \cdots & \cdots \\
			C_2 & \cdots & \cdots \\
			\vdots & \vdots & \vdots \\
			C_p & \cdots & \cdots \\
		\end{bmatrix}\begin{bmatrix}
			L_1 & L_2 & \cdots & L_n \\
			\vdots & \vdots & \vdots & \vdots \\
			\vdots & \vdots & \vdots & \vdots
		\end{bmatrix} = \begin{bmatrix}
			C_1L_1 & C_1P_2 & \cdots & C_1L_n \\
			C_2L_1 & C_2P_2 & \cdots & C_2L_n \\
			\vdots & \vdots & \vdots & \vdots \\
			C_pL_1 & C_pL_2 & \cdots & C_nL_n \\
		\end{bmatrix}= \begin{bmatrix}
			0 & 0 & \cdots & 0 \\
			0 & 0 & \cdots & 0 \\
			\vdots & \vdots & \vdots & \vdots \\
			0 & 0 & \cdots & 0 \\
		\end{bmatrix}$$
		
		\end{sol} 
		
		\item Se $Ax = b$ e $A^Ty = 0$, temos $y^Tx = 0$ ou $y^Tb=0$?
		
		\begin{sol}  
			Fazendo a transposta da equação $Ax = b$, teremos:\\
			$x^T A^T = b^T$\\
			Multiplicando a equação por y:\\
			$x^T A^T \cdot y = b^T \cdot y$\\
			$x^T (A^T y) = b^T y$\\
			Sabendo que $A^Ty = 0$:\\
			$0 = b^T y$\\
			Calculando a transposta da equação, confirmamos que:
			$y^T b = 0$
		\end{sol} 
		
		\item O sistema abaixo não tem solução:
		$$\begin{cases}
			x + 2y + 2z = 5\\
			2x + 2y + 3z = 5\\
			3x + 4y + 5z = 9
		\end{cases}$$
		Ache números $y_1,y_2,y_3$ para multiplicar as equações acima para que elas somem $0=1$. Em qual espaço fundamental o vetor $y$ pertence? Verifique que $y^Tb = 1$. O caso acima é típico e conhecido como a \textit{Alternativa de Fredholm}: ou $Ax = b$ ou $A^Ty = 0$ com $y^Tb = 1$.
		
		\begin{sol} 
			Supondo que $y_1 = 1$, $y_2 = 1$ e $y_3 = -1$, multiplicando as equações acima e depois somando:
			
			$
			\left\{
			\begin {array}{cl}
			x + 2y + 2z = 5 (1)\\
			2x + 2y + 3z = 5 (1)\\
			3x + 4y + 5z = 9 (-1)
			\end{array}
			\right.
			$\\    
			
			$
			\left\{
			\begin {array}{cl}
			x + 2y + 2z = 5 \\
			2x + 2y + 3z = 5 \\
			-3x - 4y - 5z = -9 
			\end{array}
			\right.
			$\\   
			
			$(x + 2x - 3x) + (2y + 2y - 4y) + (2z + 3z - 5z) = 5 + 5 - 9$\\
			$0 = 1$
			
			$y = (1, 1, -1)$ está no espaço nulo a esquerda, $N(A^T)$, pois:
			$$A^T \cdot y = \begin{bmatrix}
				1 & 2 & 3 \\
				2 & 2 & 4 \\
				2 & 3 & 5
			\end{bmatrix}\begin{bmatrix}
			1 \\
			1 \\
			-1
			\end{bmatrix} = \begin{bmatrix}
			1 \cdot 1 + 2 \cdot 1 + 3 \cdot (-1) \\
			2 \cdot 1 + 2 \cdot 1 + 4 \cdot (-1)\\
			2 \cdot 1 + 3 \cdot 1 + 5 \cdot (-1)\\
			\end{bmatrix} = \begin{bmatrix}
			0 \\
			0\\
			0\\
			\end{bmatrix}$$
		
			$y^T \cdot b = \begin{bmatrix}
				1 & 1 & -1 \\
			\end{bmatrix}\begin{bmatrix}
				5 \\
				5 \\
				-9
			\end{bmatrix} = \begin{bmatrix}
			1 \cdot 5 + 1 \cdot 5 - 1 \cdot - 9 
		\end{bmatrix} = 1$
			
		\end{sol} 
		
		\item Mostre que se $A^TAx = 0$, então $Ax = 0$. O oposto é obviamente verdade e então temos $N(A^TA) = N(A)$.
		
		\begin{sol} 
			Se $A^TAx = 0$, então multiplicando por $x^T$: $$x^TA^T \cdot Ax = 0$$  
			Sabendo que $x^T A^T$ é o mesmo que $(Ax)^T$, então:
								$$ (Ax)^T \cdot (Ax) = 0$$
								$$ \|Ax\|^2 = 0$$
								$$ Ax = 0$$
		\end{sol} 
		
		\item Seja $A$ uma matriz $3 \times 4$ e $B$ uma $4 \times 5$ tais que $AB = 0$. Mostre que $C(B) \subset N(A)$. Além disso, mostre que posto$(A)$ $ + $ posto$(B) \leq 4$.
		
		\begin{sol} 
			Para provar que $C(B) \subset N(A)$:
			Seja $c_b$ o conjunto de 5 matrizes $4 \times 1$ que são colunas de B, e $c_b \subset C(B)$. A partir de $AB = 0$, teremos:
			$$\begin{bmatrix}
				L_1 \\
				L_2 \\
				L_3 \\
			\end{bmatrix}\begin{bmatrix}
				C_1 & C_2 & C_3 & C_4 & C_5 \\
			\end{bmatrix} = \begin{bmatrix}
				L_1C_1 & L_1C_2 & L_1C_3 & L_1C_4 & L_1C_5 \\
				L_2C_1 & L_2C_2 & L_2C_3 & L_2C_4 & L_2C_5 \\
				L_3C_1 & L_3C_2 & L_3C_3 & L_3C_4 & L_3C_5 \\
			\end{bmatrix}= \begin{bmatrix}
				0 & 0 & 0 & 0 & 0\\
				0 & 0 & 0 & 0 & 0\\
				0 & 0 & 0 & 0 & 0\\
			\end{bmatrix}$$
			Sendo assim, $L_i \cdot c_b = 0 \Rightarrow A \cdot c_b = 0$, sendo $L_i$ as linhas de A e por isso o espaço de colunas de B está contido no espaço nulo de A, ou seja, $C(B) \subset N(A)$. E isso implica que $posto(B) = dim(col(B)) \leq dimN (A)$. 
			Pelo teorema do Posto-Nulidade:
			$$posto (B) + posto (A) \leq dimN (A) + posto (A) = n$$ 
			Neste caso, $n = 4$, então:  posto$(A)$ $ + $ posto$(B) \leq 4$.
			
		\end{sol} 
		
		\item Sejam $\mathbf{a,b,c,d}$ vetores não-zeros de $\bR^2$.
		
		\begin{enumerate}
			
			\item Quais são as condições sobre esses vetores para que cada um possa ser, respectivamente, base dos espaços $C(A^T)$, $N(A)$, $C(A)$ e $N(A^T)$ para uma dada matriz $A$ que seja $2 \times 2$. \textit{Dica: cada espaço fundamental vai ter somente um desses vetores como base.}
			
			\begin{sol} 
				Há ortogonalidade entre a base de $C(A)$ e $N(A)$ e entre $N(A^T)$ e $C(A^T)$. Então:\\
				$b \cdot c = 0$ e $a \cdot d = 0$
			\end{sol} 
			
			\item Qual seria uma matriz $A$ possível?
			
			\begin{sol} 
				  Desde que $A$ seja uma matriz de posto 1, pois as bases são vetores não-zeros, então A pode ser: $A = \begin{bmatrix}
				  	1 & 3 \\
				  	3 & 9 
				  \end{bmatrix}$
			\end{sol} 
			
		\end{enumerate}
		
		\item Ache $S^{\perp}$ para os seguintes conjuntos:
		
		\begin{enumerate}
			
			\item $S = \{0\}$
			
			\item $S = span\{[1,1,1]\}$
			
			\item $S = span\{[1,1,1], [1,1,-1]\}$
			
			\item $S = \{[1,5,1], [2,2,2]\}$. Note que $S$ não é um subespaço, mas $S^\perp$ é.
			
		\end{enumerate}
		
		\begin{sol}
			\vspace{-\baselineskip} 
			\begin{enumerate}    
				\item $S^T = E$
				
				
			--------------------------------------------------------------------------------------	 
				\item $S^T = span\{(1,0,-1),(0,1,-1)\}$\\
			Explicação:\\
			$DimS = 1$\\
			$DimS^{\perp} = 3 - 1 = 2$\\
			$(a, b, c) \in S^{\perp} \Longleftrightarrow (a, b, c)^{\perp}(1,1,1) = 0$\\
			$a + b + c = 0 \Rightarrow c = -a -b$\\
			$(a, b, c) = (a, b, -a-b) = a(1, 0, -1) + b(0, 1, -1)$
			
			--------------------------------------------------------------------------------------
				\item $S^T = span\{(1,-1,0)\}$\\
			Explicação:\\ 
			$(a,b,c) \in S^{\perp} \Rightarrow \\
			(a, b, c)^{\perp}(1,1,1) = 0$\\
			$(a, b, c)^{\perp}(1,1,-1) = 0$\\
			
			$a + b + c = 0$\\
			$a + b - c = 0$\\
			$a + b = 0$\\
			$b = -a$ e $c = 0$\\
			$(a, b, c) = (a, -a, 0) = a(1, -1, 0)$
			
			--------------------------------------------------------------------------------------								  
				\item $S^T$ está no espaço nulo de A = $\begin{bmatrix}
					1 & 5 & 1 \\
					2 & 2 & 2
				\end{bmatrix}$, então é formado por $(-1, 0, 1)$
			\end{enumerate}    
		\end{sol} 
		
		\item Seja $A$ uma matriz $4 \times 3$ formada pela primeiras 3 colunas da matriz identidade $4 \times 4$. Projeta o vetor $b = [1,2,3,4]$ no espaço coluna de $A$. Ache a matriz de projeção $P$.
		
		\begin{sol} 
		$	A = \begin{bmatrix}
				1 & 0 & 0 \\
				0 & 1 & 0 \\
				0 & 0 & 1 \\
				0 & 0 & 0
			\end{bmatrix} $
		$A^T = \begin{bmatrix}
			1 & 0 & 0 & 0\\
			0 & 1 & 0 & 0\\
			0 & 0 & 1 & 0
		\end{bmatrix}$\\
		$A^TA = \begin{bmatrix}
			1 & 0 & 0 \\
			0 & 1 & 0 \\
			0 & 0 & 1 
		\end{bmatrix} = I$\\
		$(A^TA)^{-1} = \begin{bmatrix}
			1 & 0 & 0 \\
			0 & 1 & 0 \\
			0 & 0 & 1 
		\end{bmatrix} = I$\\
		Quando há projeção em C(A): $P = A(A^TA)^{-1}A^T$, então:\\
		$P = \begin{bmatrix}
			1 & 0 & 0 \\
			0 & 1 & 0 \\
			0 & 0 & 1 \\
			0 & 0 & 0
		\end{bmatrix}\begin{bmatrix}
		1 & 0 & 0 \\
		0 & 1 & 0 \\
		0 & 0 & 1 
		\end{bmatrix}\begin{bmatrix}
		1 & 0 & 0 & 0\\
		0 & 1 & 0 & 0\\
		0 & 0 & 1 & 0
		\end{bmatrix} = A \cdot I \cdot A^T =  \begin{bmatrix}
		1 & 0 & 0 & 0\\
		0 & 1 & 0 & 0\\
		0 & 0 & 1 & 0\\
		0 & 0 & 0 & 0
		\end{bmatrix}$ \\
		Projetando o vetor b:\\
		$p = Pb = \begin{bmatrix}
			1 & 0 & 0 & 0\\
			0 & 1 & 0 & 0\\
			0 & 0 & 1 & 0\\
			0 & 0 & 0 & 0
		\end{bmatrix}\begin{bmatrix}
		1 \\
		2 \\
		3 \\
		4 
		\end{bmatrix} = \begin{bmatrix}
		1 \\
		2 \\
		3 \\
		0 
		\end{bmatrix}$
		 
		\end{sol} 
		
		\item Se $P^2 = P$, mostre que $(I - P)^2 = I - P$. Para a matriz $P$ do exercício anterior, em qual subespaço a matriz $I - P$ projeta?
		
		\begin{sol} 
			$(I-P)^2 = (I - P)(I - P) = I - PI - PI + P^2 = I - P - P + P^2$,\\
			Substituindo $P^2 = P$, obtemos:\\
			$= I -P \cancel{-P} \cancel{+P} = I - P $ \\
			
			A partir de:\\
			$(P\vec{x})\cdot((I - P)\vec{x})$\\
			$(P\vec{x})^T\cdot(I - P)\vec{x} = \vec{x}^T P^T(I - P)\vec{x}$\\
			Visto que $P^T = P$ pois é simétrica:\\
			$\vec{x}^T P(I - P)\vec{x}$\\
			$\vec{x}^T (P - P^2)\vec{x}$\\
			Visto que $P^2 = P$:\\
			$\vec{x}^T (P - P)\vec{x}$\\
			$\vec{x}^T 0 \vec{x} = 0$\\
			O produto entre $(P\vec{x})^T$ e $(I - P)\vec{x}$ é 0, então espera-se ortogonalidade entre $P\vec{x}$ e $(I - P)\vec{x}$.\\
			$P\vec{x}$ está em $C(P)$, enquanto $(I - P)\vec{x}$ está em $N(P^T)$.
			Por isso a matriz $(I - P)$ projeta no espaço nulo a esquerda.
		\end{sol} 
	\end{enumerate}
	
	
	
	
	
	
	
	
	
	
	
	
	
	
	
\end{document} 