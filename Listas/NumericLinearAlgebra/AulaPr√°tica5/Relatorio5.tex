\documentclass[leqno]{article}
\usepackage[utf8x]{inputenc}
\usepackage{float}
\usepackage[brazil]{babel} %\usepackage[latin1]{inputenc}
\usepackage{a4wide}
\usepackage{mathtools}
\usepackage{nccmath}
\setlength{\oddsidemargin}{-0.2in}
% % \setlength{\oddsidemargin}{0.2in}
\setlength{\evensidemargin}{-0.2in}
% % \setlength{\evensidemargin}{0.5in}
% % \setlength{\textwidth}{5.5in}
\setlength{\textwidth}{6.5in}
\setlength{\topmargin}{-1.2in}
\setlength{\textheight}{10in}
\usepackage[]{amsfonts} \usepackage[]{amsmath}
\usepackage[]{amssymb} \usepackage[]{latexsym}
\usepackage{graphicx,color} \usepackage{amsthm}
\usepackage{mathrsfs} \usepackage{url}
\usepackage{cancel} \usepackage{enumerate}
\usepackage{xifthen} \usepackage{tikz}
\usetikzlibrary{automata,arrows,positioning,calc}
\usepackage{listings}
\usepackage{tcolorbox}
\numberwithin{equation}{section}

\setlength{\parindent}{12 pt}

\definecolor{codegray}{rgb}{0.9, 0.9, 0.9}
\definecolor{codepurple}{rgb}{0.58,0,0.82}
\definecolor{codegreen}{rgb}{0,0.6,0}

\lstdefinestyle{mystyle}{
	language=Scilab,
	backgroundcolor=\color{codegray},   
	commentstyle=\color{codegreen},
	keywordstyle=\color{magenta},
	emph={testfunc,print,src},
	emphstyle=\color{codeyellow},
	numberstyle=\tiny\color{codegray},
	stringstyle=\color{codepurple},
	basicstyle=\footnotesize,
	breakatwhitespace=false,         
	breaklines=true,                 
	captionpos=b,                    
	keepspaces=true,                 
	numbers=left,                    
	numbersep=5pt,                  
	showspaces=false,                
	showstringspaces=false,
	showtabs=false,                  
	tabsize=2
}

\lstset{style=mystyle}

\begin{document}
	
	\newtheorem{teo}{Teorema}[section] \newtheorem*{teo*}{Teorema}
	\newtheorem{prop}[teo]{Proposição} \newtheorem*{prop*}{Proposição}
	\newtheorem{lema}[teo]{Lemma} \newtheorem*{lema*}{Lema}
	\newtheorem{cor}[teo]{Corolário} \newtheorem*{cor*}{Corolário}
	
	\theoremstyle{definition}
	\newtheorem{defi}[teo]{Definição} \newtheorem*{defi*}{Definição}
	\newtheorem{exem}[teo]{Exemplo} \newtheorem*{exem*}{Exemplo}
	\newtheorem{obs}[teo]{Observação} \newtheorem*{obs*}{Observação}
	\newtheorem*{hipo}{Hipóteses}
	\newtheorem*{nota}{Notação}
	
	\newcommand{\ds}{\displaystyle} \newcommand{\nl}{\newline}
	\newcommand{\eps}{\varepsilon} \newcommand{\ssty}{\scriptstyle}
	\newcommand{\bE}{\mathbb{E}}
	\newcommand{\cB}{\mathcal{B}}
	\newcommand{\cF}{\mathcal{F}}
	\newcommand{\cA}{\mathcal{A}}
	\newcommand{\cM}{\mathcal{M}}
	\newcommand{\cD}{\mathcal{D}}
	\newcommand{\cN}{\mathcal{N}}
	\newcommand{\cL}{\mathcal{L}}
	\newcommand{\cLN}{\mathcal{LN}}
	\newcommand{\bP}{\mathbb{P}}
	\newcommand{\bQ}{\mathbb{Q}}
	\newcommand{\bN}{\mathbb{N}}
	\newcommand{\bR}{\mathbb{R}}
	\newcommand{\bZ}{\mathbb{Z}}
	
	\newcommand{\bfw}{\mathbf{w}}
	\newcommand{\bfv}{\mathbf{v}}
	\newcommand{\bfu}{\mathbf{u}}
	\newcommand{\bfx}{\mathbf{x}}
	\newcommand{\bfb}{\mathbf{b}}
	
	\newcommand{\bvecc}[2]{%
		\begin{bmatrix} #1 \\ #2  \end{bmatrix}
	}
	\newcommand{\bveccc}[3]{%
		\begin{bmatrix} #1 \\ #2 \\ #3  \end{bmatrix}
	}
	
	\newenvironment{sol} 
	{
		\vspace{4mm}
		\noindent\textbf{{\large Código:}}
		\strut
		\smallskip
		\hspace{-3.5mm} 
	} 
	% Objetos que aparecem *após* o ambiente. 
	% (você pode, por exemplo, modificar, 
	% ou remover, a barra horizontal} 
	%{\noindent\rule{4cm}{.1mm}}
	
	
	\title{Álgebra Linear - Aula Prática 5}
	
	\author{Iara Cristina Mescua Castro}
	
	\date{\today}
	
	\maketitle 
	
		\noindent\textbf{{\large MÉTODO DE GRAM-SCHMIDT}}\\
		
		\begin{tcolorbox}[colback=green!5,colframe=green!40!black]
			Escreva uma função Scilab function [Q,R] = qr\_GS(A) que implementa o Método de Gram-Schmidt para determinar a decomposição QR de uma matriz A com colunas linearmente independentes.
			Testar a sua função com algumas matrizes de ordens diferentes. Para cada uma delas, testar a precisão do método (por exemplo, teste a ortogonalidade da matriz Q obtida calculando QTQ).
		\end{tcolorbox}
	
		\noindent A decomposição QR pelo método de Gram-Schmidt funciona da seguinte forma: 
		$$A_{m x n} =  Q_{m x n}R_{n x m}$$
		$$\begin{bmatrix}
			a_1 & a_2 & \cdots & a_n
		\end{bmatrix} = \begin{bmatrix}
			q_1 & q_2 & \cdots & q_n
		\end{bmatrix}\begin{bmatrix}
			r_{11} & r_{12} & r_{13} & \cdots & r_{1m} \\
			0 & r_{22} & r_{23} & \cdots & r_{2m} \\
			0 & 0 & r_{33} & \cdots & r_{3m} \\
			0 & 0 & 0 & \cdots & r_{mm}
		\end{bmatrix}$$\\
		
		\vspace{0.3cm}
		\noindent Sabendo que $r_{ij}$:\\
		$0$, se $i>j$\\
		$||v_{j}||$, se $i = j$\\
		$q_{i}^T a_j$, se $i < j$
		
		\vspace{0.3cm}
		\noindent Então:\\
		$a_1 = r_{11} q_{1}$\\
		$a_2 = r_{12} q_{1} + r_{22} q_{2}$\\
		$a_n = r_{1m} q_{1} + r_{2m} q_{2} + r_{3m} q_{3} + \cdots + r_{mm} q_{n}$
		
		\vspace{0.3cm}
		\noindent É o mesmo que:\\
		$a_1 = ||v_1||q_1$\\
		$a_2 = (q_{1}^T a_2) q_1 + ||v_2||q_2$\\
		$a_3 = (q_{1}^T a_3) q_1 + (q_{2}^T a_3)q_2 + ||v_3||q_3$

		\begin{sol}
		\begin{lstlisting}[style=mystyle, language=Scilab]
			function [Q,R] = qr_GS(A)
			[m,n] = size(A); // le dim de A
			// inicializa as matrizes Q e R
			Q = zeros(m,n);
			R = zeros(n,m);
			for j = 1:n 
				v = A(:,j);
				for i = 1:(j-1)
					R(i,j) = Q(:,i)' * A(:,j); // r_ij caso i < j
					v = v - R(i,j)*Q(:,i);
				end
				R(j,j) = norm(v, 2); // r_ij, quando i = j
				Q(:,j) = v/(R(j,j));
			end
			endfunction
		\end{lstlisting}	
	
	\vspace{0.3cm}	
	\end{sol}

	Para os testes, criei uma função que gera uma matriz aleatória quadrada de ordem n, e uma que gera uma matriz aleatória (m,n):
	\begin{lstlisting}[style=mystyle, language=Scilab]
		function [A] = matriz_aleat(n) //quadrada
		//para gerar um numero aleatorio entre [a,b]
		//  r = a + (b-a)*rand();
		
		//matriz A(m,n)) com numeros aleatorios entre [1 e n^3]
		A = floor(1 + ((n^3 - 1)*rand(n,n,'uniform')))
		endfunction
		
		function [A] = matriz_aleat2(n,m) 
		//para gerar um numero aleatorio entre [a,b]
		//  r = a + (b-a)*rand();
		
		//matriz A(m,n)) com numeros aleatorios entre [1 e n^3]
		A = floor(1 + ((n^3 - 1)*rand(n,m,'uniform')))
		endfunction		
	\end{lstlisting}

	\noindent Testando com o \textbf{Método de Gram-Schmidt}:\\

	\noindent \textbf{{\large ORDEM 5}}
	
	Exemplo 1:
	
	\begin{figure}[H]
		\centering
		\includegraphics[width=0.6\linewidth]{figures/a1_gs}		
	\end{figure}

	Sabendo que Q é uma matriz ortogonal, logo, $Q^TQ = I$. Então podemos testar sua ortogonalidade vendo o quão próximo esse produto com nosso Q se aproxima da matriz identidade.
	
	\begin{figure}[H]
		\centering
		\includegraphics[width=0.6\linewidth]{figures/a1_gs_orto}		
	\end{figure}

	A diagonal é composta por 1's, e os demais índices estão muito próximos de 0 em cerca de 15 e 16 casas decimais.


	Observação: Vale ressaltar que mesmo os números inteiros do produto QR sendo os mesmos, o produto não necessariamente é igual. Isso fica evidente ao realizar $Q*R - A$, que deveria ser 0, mas ao testar em um exemplo qualquer não é totalmente nulo.
	
	\begin{figure}[H]
		\centering
		\includegraphics[width=1\linewidth]{figures/a5_gsm}		
	\end{figure}

	
	Exemplo 2:
	
	\begin{figure}[H]
		\centering
		\includegraphics[width=0.6\linewidth]{figures/a2_gs}
	\end{figure}

	\begin{figure}[H]
		\centering
		\includegraphics[width=0.6\linewidth]{figures/a2_gs_orto}		
	\end{figure}

	A diagonal é composta por 1's, e os demais índices estão muito próximos de 0 em cerca de 16 e 17 casas decimais.

	
	--------------------------------------------------------------------------------------------------------------------------------------\\
	
	\textbf{{\large ORDEM 10}}
	
	Exemplo 1:
	
	\begin{figure}[H]
		\centering
		\includegraphics[width=0.6\linewidth]{figures/a3_gs}		
	\end{figure}

	
	Testando sua ortogonalidade:
	
	\begin{figure}[H]
		\centering
		\includegraphics[width=0.8\linewidth]{figures/a3_gs_orto}		
	\end{figure}
	\newpage
	Exemplo 2:
	
	\begin{figure}[H]
		\centering
		\includegraphics[width=0.6\linewidth]{figures/a4_gs}		
	\end{figure}
	
	\begin{figure}[H]
		\centering
		\includegraphics[width=1\linewidth]{figures/a4_gs_orto}		
	\end{figure}


		--------------------------------------------------------------------------------------------------------------------------------------\\
		\newpage
		\noindent \textbf{{\large Matrizes Não-Quadradas}}
		
		Exemplo 1: A(m,n), onde $n > m$ Mais colunas do que linhas.
		
		\begin{figure}[H]
			\centering
			\includegraphics[width=0.5\linewidth]{figures/gs_nao_quadradas_1}		
		\end{figure}
	
		Como esperado, a matriz Q tem tamanho (m,n) e a matriz R tem tamanho (n,n).
	
		Exemplo 2: A(m,n), onde $m > n$ Mais linhas do que colunas.
		
		\begin{figure}[H]
			\centering
			\includegraphics[width=0.5\linewidth]{figures/gs_nao_quadradas_2}		
		\end{figure}
	
		Como esperado, a matriz Q tem tamanho (m,n) e a matriz R tem tamanho (n,n).
		
		--------------------------------------------------------------------------------------------------------------------------------------\\
	
		\noindent\textbf{{\large	2) MÉTODO DE GRAM-SCHMIDT MODIFICADO}}
		
	\begin{tcolorbox}[colback=green!5,colframe=green!40!black]
		Escreva uma função Scilab function [Q,R] = qr\_GSM(A) que implementa o Método de Gram-Schmidt Modificado.
		Testar a sua função com as mesmas matrizes usadas nos testes do item anterior. Comparar a precisão dos dois Métodos.
	\end{tcolorbox}

	\noindent Para o método modificado, vamos usar v em vez de $a_j$ para $r_ij$ quando i < j:
	
	\begin{sol}			
		\begin{lstlisting}[style=mystyle, language=Scilab]
			function [Q,R] = qr_GSM(A)
			[m,n] = size(A); // le dim de A
			//inicializa as matrizes Q e R
			Q = zeros(m,n);
			R = zeros(n,n);
			//loop
			for j = 1:n
				v = A(:,j);
				for i = 1:(j-1)
					R(i,j) = Q(:,i)' * v; // r_ij caso i < j
					v = v - R(i,j)*Q(:,i);
				end
				R(j,j) = norm(v, 2); // r_ij, quando i = j
				Q(:,j) = v/(R(j,j));
			end
			endfunction
		\end{lstlisting}
	\end{sol}

	\noindent Agora a função é mais estável em relação a ortogonalidade.
	\vspace{3mm}
	
	Testando com o \textbf{Método de Gram-Schmidt Modificado}:\\
	
	\textbf{{\large ORDEM 5}}
	
	Exemplo 1:
	
	\begin{figure}[H]
		\centering
		\includegraphics[width=0.6\linewidth]{figures/a1_gsm}		
	\end{figure}
	
	Assim como antes, podemos testar sua ortogonalidade vendo o quão próximo $Q^TQ = I$ com nosso Q se aproxima da matriz identidade.
	
	\begin{figure}[H]
		\centering
		\includegraphics[width=0.6\linewidth]{figures/a1_gsm_orto}		
	\end{figure}

	Observação: Vale ressaltar que aqui também, mesmo os números inteiros sendo os mesmos, o produto não necessariamente é igual. Isso fica evidente ao realizar $Q*R - A$, que deveria ser 0, mas assim como na GS, não é totalmente.
	
	\begin{figure}[H]
		\centering
		\includegraphics[width=1\linewidth]{figures/a5_gsm}		
	\end{figure}
	
	Exemplo 2:
	
	\begin{figure}[H]
		\centering
		\includegraphics[width=0.8\linewidth]{figures/a2_gsm}
	\end{figure}
	
	\begin{figure}[H]
		\centering
		\includegraphics[width=0.6\linewidth]{figures/a2_gsm_orto}		
	\end{figure}

	\textbf{{\large ORDEM 10}}
	
	Exemplo 1:
	
	\begin{figure}[H]
		\centering
		\includegraphics[width=0.6\linewidth]{figures/a3_gsm}		
	\end{figure}
	
	
	Testando sua ortogonalidade:
	
	\begin{figure}[H]
		\centering
		\includegraphics[width=1\linewidth]{figures/a3_gsm_orto}		
	\end{figure}
	\newpage
	Exemplo 2:
	
	\begin{figure}[H]
		\centering
		\includegraphics[width=0.6\linewidth]{figures/a4_gsm}		
	\end{figure}
	
	\begin{figure}[H]
		\centering
		\includegraphics[width=1\linewidth]{figures/a4_gsm_orto}		
	\end{figure}

	--------------------------------------------------------------------------------------------------------------------------------------\\
	\newpage
	\noindent \textbf{{\large Matrizes Não-Quadradas}}
	
	Exemplo 1: A(m,n), onde $n > m$ Mais colunas do que linhas.
	
	\begin{figure}[H]
		\centering
		\includegraphics[width=0.5\linewidth]{figures/gsm_nao_quadradas_1}		
	\end{figure}
	
	Como esperado, a matriz Q tem tamanho (m,n) e a matriz R tem tamanho (n,n).
	
	Exemplo 2: A(m,n), onde $m > n$ Mais linhas do que colunas.
	
	\begin{figure}[H]
		\centering
		\includegraphics[width=0.5\linewidth]{figures/gsm_nao_quadradas_2}		
	\end{figure}
	
	Como esperado, a matriz Q tem tamanho (m,n) e a matriz R tem tamanho (n,n).
	\newpage
	\noindent \textbf{{\large	4) MÉTODO DE HOUSEHOLDER}}
	
	\begin{tcolorbox}[colback=green!5,colframe=green!40!black]
	Escreva uma função Scilab function [U,R] = qr\_House(A) que implementa o Método de Householder para determinar a decomposição QR de uma matriz A. A matriz U, triangular inferior, deve conter em suas colunas os vetores unitários que geraram as matrizes dos refletores de Householder usadas para gerar a decomposição QR.
	Escreva também uma função Scilab function [Q] = constroi\_Q\_House(U) que constrói a matriz ortogonal Q da decomposição A = QR a partir da matriz U retornada pela função function [U,R] = qr\_House(A).
	\end{tcolorbox}
		
	No método de Householder, vamos decompor A, mas nossa função inicialmente retornará as matrizes U, R em vez de Q e R.
	
	Basicamente, vamos criar um loop para transformar A na matriz R da seguinte forma: \\
	$$A = R_0$$
	$$Q_1 A = Q_1 R = R_1$$
	$$Q_2 Q_1 A = Q_2 R_1 = R_2$$
	$$\vdots$$
	$$Q_j Q_{j-1} \cdots Q_1 A = Q_j R_{j-1} = R_j$$
	$$\vdots$$
	$$Q_k \cdots Q_1 A = R$$
	
	Para a reconstrução de Q, basta lembrar que:
	
	$$Q_k \cdots Q_2 Q_1 A = R$$
	$$= Q^TA = R$$
	$$A = QR$$
	$$Q = (Q_k \cdots Q_2 Q_1)^T$$
	$$Q = Q_1^T Q_2^T \cdots Q_k^T$$
	$$Q = Q_1 Q_2 \cdots Q_k$$
	
	\noindent Então vamos criar um loop para multiplicar as matrizes Q pelo refletor de Householder: $I - 2uu^T$\\	
	Juntando as duas funções:
	
	\begin{sol}			
		\begin{lstlisting}[style=mystyle, language=Scilab]
			function [Q, U, R] = Householder(A)
			[m,n] = size(A);  //Ler dim de A
			k = min(m-1, n); //Tamanho de U depende do tamanho de A
			for j = 1:k
				x = A(j:m, j); //Coluna j da linha j ate a final
				if x(1) > 0 then
					x(1) = x(1) + norm(x, 2);
				else
					x(1) = x(1) - norm(x,2);
				end
			
				u = x/norm(x,2);
				U(j:m, j) = u; //Guarda o vetor u em U
			
				I = eye(m - j + 1);
				H = I - 2*u*u';
				A(j:m, j:m) = A(j:m, j:m) - 2*u*u'*A(j:m, j:m);
			end
			
			R = triu(A) // R e a matriz A alterana apos varias iteracoes
			//limpa os zeros
			Q = constroi_Q(U); // Constroi Q a partir de U
			endfunction
			
			function Q = constroi_Q(U) 
			[m,k] = size(U);
			Q = eye(m,m);
			for j = 1:k
				u = U(1:m, j)
				//Q = Q(I - 2uu^T)
				Q = Q - 2*(Q*u)*u'; 
			end
			endfunction
		\end{lstlisting}
	\end{sol}

	Testando com o \textbf{Método de Householder}:
	
	
	\textbf{{\large ORDEM 5}}
	
	Exemplo 1:
	
	\begin{figure}[H]
		\centering
		\includegraphics[width=0.6\linewidth]{figures/a1_house_1}		
	\end{figure}

	Curiosamente, ao usar a função qr(A) do próprio scilab, é possível observar que a decomposição retorna as matrizes exatamente iguais às da nossa função de Householder.
	
	\begin{figure}[H]
		\centering
		\includegraphics[width=0.6\linewidth]{figures/a1_house_2}		
	\end{figure}

	Observação: Vale ressaltar que mesmo os números inteiros sendo os mesmos, o produto não necessariamente é igual. Isso fica evidente ao realizar $Q*R - A$, que deveria ser 0, mas assim como nos outros casos, não é totalmente.
	
	\begin{figure}[H]
		\centering
		\includegraphics[width=0.6\linewidth]{figures/a5_house}		
	\end{figure}
	
	Assim como antes, podemos testar sua ortogonalidade vendo o quão próximo $Q^TQ = I$ com nosso Q se aproxima da matriz identidade.
	
	\begin{figure}[H]
		\centering
		\includegraphics[width=0.6\linewidth]{figures/a1_house_orto}		
	\end{figure}
	
	Exemplo 2:
	
	\begin{figure}[H]
		\centering
		\includegraphics[width=0.6\linewidth]{figures/a2_house_1}
	\end{figure}

	\begin{figure}[H]
		\centering
		\includegraphics[width=0.6\linewidth]{figures/a2_house_2}
	\end{figure}
	
	\newpage
	\textbf{{\large ORDEM 10}}
	
	Exemplo 1:
	
	\begin{figure}[H]
		\centering
		\includegraphics[width=0.6\linewidth]{figures/a3_house}		
	\end{figure}
	
	
	Testando sua ortogonalidade:
	
	\begin{figure}[H]
		\centering
		\includegraphics[width=1\linewidth]{figures/a3_house_orto}		
	\end{figure}

	Exemplo 2:
	
	\begin{figure}[H]
		\centering
		\includegraphics[width=0.6\linewidth]{figures/a4_house}		
	\end{figure}
	
	
	Testando sua ortogonalidade:
	
	\begin{figure}[H]
		\centering
		\includegraphics[width=1\linewidth]{figures/a4_house_orto}		
	\end{figure}

	\newpage
	\noindent \textbf{{\large Matrizes Não-Quadradas}}
	
	Exemplo 1: A(m,n), onde $n > m$ Mais colunas do que linhas, então k = m - 1;
	
	\begin{figure}[H]
		\centering
		\includegraphics[width=0.5\linewidth]{figures/house_nao_quadradas_1}		
	\end{figure}
	
	Como esperado, a matriz Q é (m,m), (4,4). U é (m, m-1), (4,3). R tem tamanho (m,n), ou seja, (4,5).
	
	Exemplo 2: A(m,n), onde $m > n$ Mais linhas do que colunas, então k = n.
	
	\begin{figure}[H]
		\centering
		\includegraphics[width=0.5\linewidth]{figures/house_nao_quadradas_2}		
	\end{figure}

	\begin{figure}[H]
		\centering
		\includegraphics[width=0.3\linewidth]{figures/house_nao_quadradas_3}		
	\end{figure}
	
	Como esperado, a matriz Q é (m,m), (5,5). U é (m, n), (5,4). R tem tamanho (m,n), ou seja, (5,4).
	
	--------------------------------------------------------------------------------------------------------------------------------------\\

	\noindent \textbf{{\large 4.1) Testar as suas funções com as mesmas matrizes usadas nos testes dos itens
	anteriores. Comparar a precisão dos Métodos.}}
	
	
	
	Para comparar as funções é mais conveniente o uso de matrizes com ordem grande, e podemos conferir sem precisar abrir as matrizes. Criei uma função que faz a a norma dos erros para $$Q*R - A$$
	
	\begin{lstlisting}[style=mystyle, language=Scilab]
		function erro = media_erro_qr(A) //QR - A -> 0
		[m,n] = size(A);
		erro = abs(norm(A,2))/ (m*n)
		endfunction
	\end{lstlisting}
	
	Analogamente, podemos calcular a média dos erros de $$Q'*Q - I$$ com:
	
	\begin{lstlisting}[style=mystyle, language=Scilab]
		function erro = media_erro_qq(A) //Q'Q - I -> 0
		[m,n] = size(A);
		A = A - eye(m,n);
		erro = abs(norm(A),2)/ (m*n)
		endfunction
	\end{lstlisting}
	
	Primeiro subtraímos A da matriz identidade para obter os resíduos, e então calculamos a média dividindo sua soma pelo total de índices $m*n$.\\
	
	\noindent \textbf{{\large ORDEM 50}}
	
	
	\noindent Erro QR\\ \noindent Exemplo 1:	
	\begin{figure}[H]
		\centering
		\includegraphics[width=0.5\linewidth]{figures/erros1}		
	\end{figure}

	\noindent Erro QR\\ \noindent Exemplo 2:
	\begin{figure}[H]
		\centering
		\includegraphics[width=0.6\linewidth]{figures/erros1.2}		
	\end{figure}

	\noindent Erro QR\\ \noindent Exemplo 3:
	\begin{figure}[H]
		\centering
		\includegraphics[width=0.6\linewidth]{figures/erros1.3}		
	\end{figure}

	\noindent \textbf{{\large ORDEM 100}}
	
	
	\noindent Erro QR\\ \noindent Exemplo 1:
	\begin{figure}[H]
		\centering
		\includegraphics[width=0.6\linewidth]{figures/erros1.4}		
	\end{figure}

	\newpage
	\noindent \textbf{{\large ORDEM 50}}

	
	\noindent Erro QQ\\ \noindent Exemplo 1:
	\begin{figure}[H]
		\centering
		\includegraphics[width=0.55\linewidth]{figures/erros2}		
	\end{figure}	

	\noindent Exemplo 2:

	\begin{figure}[H]
		\centering
		\includegraphics[width=0.5\linewidth]{figures/erros3}		
	\end{figure}

	\noindent \textbf{{\large ORDEM 100}}
	
	\begin{figure}[H]
		\centering
		\includegraphics[width=0.5\linewidth]{figures/erros4}		
	\end{figure}
	
	Todos os exemplos tiveram boa precisão. O método de Householder obtém um Q com maior ortogonalidade, visto que tem menor erro para $Q'Q - I$ mas o método de Gram-Smitch se mostra superior para o produto $Q*R$ que se aproxima mais de A.\\

	\newpage
	\textbf{{\large 4.2) Testar as suas funções com a matriz A = [0,70000 0,70711; 0,70001 0,70711].
	Comparar a ortogonalidade das matrizes Q produzidas pelos Métodos.}}

	\begin{figure}[H]
		\centering
		\includegraphics[width=0.5\linewidth]{figures/4.21}		
	\end{figure}

	\begin{figure}[H]
		\centering
		\includegraphics[width=0.5\linewidth]{figures/4.21}		
	\end{figure}

	Tanto a função GS quando a GSM tiveram o mesmo erro de 11 casas decimais.

	\begin{figure}[H]
		\centering
		\includegraphics[width=0.4\linewidth]{figures/4.23}		
	\end{figure}

	A função de Householder teve o melhor desempenho, alcançando uma ortogonalidade perfeita neste exemplo.


	\noindent \textbf{{\large 4.3) Seja a matriz $$A = \begin{bmatrix}
				1 & 2 & 3 \\
				4 & 5 & 6 \\
				7 & 8 & 7 \\
				4 & 2 & 3 \\
				4 & 2 & 2
			\end{bmatrix}$$ Calcule a decomposição QR (reduzida) usando os
	métodos de Gram-Schmidt, Householder e a função “qr” do Scilab. Compare os três
	resultados. Comente.}}
	
	\noindent \textbf{{\large Testando a função de Gram-Schmidt:}}
	\begin{figure}[H]
		\centering
		\includegraphics[width=0.5\linewidth]{figures/4.31}		
	\end{figure}

	\begin{figure}[H]
		\centering
		\includegraphics[width=0.4\linewidth]{figures/4.32}		
	\end{figure}

	\noindent \textbf{{\large Testando a função de Gram-Schmidt Modificada:}}
	\begin{figure}[H]
		\centering
		\includegraphics[width=0.5\linewidth]{figures/4.33}		
	\end{figure}

	\begin{flushleft}
		\begin{figure}[H]
		\centering
		\includegraphics[width=0.4\linewidth]{figures/4.34}		
	\end{figure}
	\end{flushleft}
	\newpage
	\textbf{{\large Testando a função de Householder:}} 
	
	\begin{figure}[H]
		\centering
		\includegraphics[width=0.65\linewidth]{figures/4.35}		
	\end{figure}

	\begin{figure}[H]
		\centering
		\includegraphics[width=0.6\linewidth]{figures/4.36}		
	\end{figure}
	\newpage
	\textbf{{\large Testando a função qr do Scilab:}}

	\begin{figure}[H]
		\centering
		\includegraphics[width=0.6\linewidth]{figures/4.37}		
	\end{figure}

	\begin{figure}[H]
		\centering
		\includegraphics[width=0.5\linewidth]{figures/4.38}		
	\end{figure}

	\textbf{{\large Comentários: }}
	
	Assim como nos testes, a função de Grand-Smitch modificada teve uma leve melhora de desempenho na ortogonalidade de Q em comparação com o método original. Visto que o erro se encontra em cerca de 17 e até 18 casas decimais. Além disso, também mostramos novamente que a função QR do Scilab utiliza o método de Householder, visto que as matrizes obtidas na decomposição são exatamente iguais, e consequentemente sua precisão em ortogonalidade também.\\
	
	\textbf{{\large	5) ALGORITMO QR para AUTOVALORES}}
	
	\begin{tcolorbox}[colback=green!5,colframe=green!40!black]
		Escreva uma função Scilab function [S] = espectro(A, tol) que calcula os autovalores
		de uma matriz simétrica A usando o Algoritmo QR. Os autovalores calculados devem
		ser devolvidos no vetor S. Use como critério de parada a norma infinito da diferença
		entre dois espectros consecutivos menor do que uma tolerância tol dada ($10^{-3}$, $10^{-4}$,
		$10^{-5}$, ...). Teste a sua função com matrizes simétricas das quais você saiba quais são
		os autovalores.
	\end{tcolorbox}

	Para essa função, vamos usar a função de HouseHolder para encontrar a decomposição QR da nossa matriz A.
	Depois, vamos ter um $A_1$, tal que $A_1 = R*Q$. Essa troca ocorre pois:
	
	$$A = QR$$
	$$R = Q^T A$$
	$$A_1 = Q^T A Q$$
	
	Feito isso, podemos concluir que $A_1$ e $A$ são semelhantes, ou seja, tem os mesmos autovalores. A partir daí iremos continuar gerando matrizes A com $RQ$ de decomposições $QR$ até obter um $R_{k-1}$ que converge para uma matriz triangular superior, cujo os autovalores serão a diagonal de $A_k$.

	\begin{sol}			
		\begin{lstlisting}[style=mystyle, language=Scilab]
			function [S] = espectro(A, tol)
			
			erro = 1;
			while erro > tol
			[Q, U, R] = Householder(A);
			A1 = R*Q;
			[Q1, U1, R1] = Householder(A1);
			A = R1*Q1;
			
			//definir erro baseado nas diagonais convergindo para os autovalores
			erro = abs(norm(diag(A), 'inf') - norm(diag(A1),'inf'));
			end
			S = diag(A1);
			endfunction
		\end{lstlisting}
	\end{sol}

	Para os testes, vou usar uma função de gerar matrizes simétricas aleatórias usada na AP3:
	
	\begin{lstlisting}
		function A = Matriz_Simetrica_Aleat(n)
		// gera numeros aleatorios entre -n^2 e n^2  
		A = floor(-((n^2)*rand(n,n,'uniform')) + ((n^2)*rand(n,n,'uniform')))
		// define a simetria da matriz
		A = tril(A) + triu(A', 1)
		endfunction
	\end{lstlisting}

	\textbf{{\large Ordem 5:}}
	
	\begin{figure}[H]
		\centering
		\includegraphics[width=0.5\linewidth]{figures/5.1}		
	\end{figure}

	\begin{figure}[H]
		\centering
		\includegraphics[width=0.4\linewidth]{figures/5.2}		
	\end{figure}

	Com essa tolerância os autovalores já alcançam seus respectivos valores. Note que a função retorna eles de acordo com sua posição na matriz, enquanto o spec() do scilab retorna eles em ordem crescente.\\


	\textbf{{\large Ordem 10:}}
	
	\begin{figure}[H]
		\centering
		\includegraphics[width=0.5\linewidth]{figures/5.3}		
	\end{figure}
	
	Mas sabendo que a função spec() do scilab ordena os autovalores, podemos compará-los lado a lado ordenando nossos autovalores com a função gsort()
	
	\begin{figure}[H]
		\centering
		\includegraphics[width=0.3\linewidth]{figures/5.4}		
	\end{figure}

	Confirmamos que a precisão da função está boa, sendo igual em todas as casas decimais da visualização.	
	
	\textbf{{\large Ordem 15:}}
	
	\begin{figure}[H]
		\centering
		\includegraphics[width=0.6\linewidth]{figures/5.5}		
	\end{figure}

	\textbf{{\large Ordem 50:}}
	Para confirmar a precisão em uma ordem maior, sem ter que abrir as matrizes, basta utilizar a mesma lógica dos testes anteriores e pegar o módulo da subtração de ambos para verificar a diferença.
	
	\begin{figure}[H]
		\centering
		\includegraphics[width=0.5\linewidth]{figures/5.6}		
	\end{figure}

	O erro foi relativamente baixo.

\end{document}